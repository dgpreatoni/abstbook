\documentclass[final,italian]{memoir}
%%%%%%%%%%%%%%%%%%%%%%%%%%%%%%%%%%%%%%%%%%%%%%%%%%%%%%%%%%%%%%%%%%%%%%%%%%%%%%%
%% Hystrix, the Italian Journal of Mammalogy
%%
%% LaTeX source file to generate an issue front matter pages
%%
%% version 1.0
%% created by prea@uninsubria.it 20120525
%%
%%%%%%%%%%%%%%%%%%%%%%%%%%%%%%%%%%%%%%%%%%%%%%%%%%%%%%%%%%%%%%%%%%%%%%%%%%%%%%%
%% This program is free software: you can redistribute it and/or modify
%% it under the terms of the GNU General Public License as published by
%% the Free Software Foundation, either version 3 of the License, or
%% (at your option) any later version.
%%
%% This program is distributed in the hope that it will be useful,
%% but WITHOUT ANY WARRANTY; without even the implied warranty of
%% MERCHANTABILITY or FITNESS FOR A PARTICULAR PURPOSE.  See the
%% GNU General Public License for more details.
%%
%% You should have received a copy of the GNU General Public License
%% along with this program.  If not, see <http://www.gnu.org/licenses/>.
%%%%%%%%%%%%%%%%%%%%%%%%%%%%%%%%%%%%%%%%%%%%%%%%%%%%%%%%%%%%%%%%%%%%%%%%%%%%%%%
%% Notes
%%
%% This LaTeX source automatically generates the front pages for an Hystrix issue.
%% Front pages consists of:
%% - a guard page, resembling the cover, which is automatically generated by the macro
%%   \hysguardpage. This page in some cases can show an issue title (if any), this is
%%   done using the \hysguard title command (see further in the file, its usage is described 
%%   in detail where tha command appears). Similarly, the command \hysguardeditor typesets
%%   the phrase "edited by" along with the names of any Editors.
%% - a colophon page (i.e. the verso of the guard page), automatically generated
%%   by \hyscolophon. Should data appearing on this page be changed, change the
%%   \hyscolophon macro in hystrixmacros.tex
%% - any other page which does not belog to an article, such as a dedication page,
%%   a preface, or tha Journal "News" section. These pages vary enormuosly and must
%%   be typesetted manually for a specific issue.l
%%
%% To prepare an issue frontmatter:
%% 1 get a copy of this file and place it into the directory containing all the 
%%   needed files to produce the issue.
%% 2 type in values for the following variables:
%% elements to build automatically the frontmatter details
\newcommand{\volume}{}      % place here the volume number
\newcommand{\issue}{}        % place here the issue number
\newcommand{\volyear}{2015}   % place here the volume year
\newcommand{\ISSN}{} % place here the journal ISSN
\newcommand{\printedmonth}{settembre}  % place here the printing date month 
\newcommand{\printedyear}{\volyear} % place here the printing date year
\newcommand{\printedpress}{} % place here the printer's name

\usepackage{xcolor}
\usepackage{afterpage}

%% redefine some fonts
\usepackage[sfdefault]{FiraSans}

\usepackage{tcolorbox}

\usepackage{pdfpages}

\include{gircmacros}

%% the document begins here
\begin{document}

%% start typesetting front matter here
\pagestyle{empty}

%% Issue title: if this issue has a title, use the \hysguardtitle command below,
%% uncommenting it and writing the issue title between the braces
%\hysguardtitle{}

%% Issue editor(s): if this issue has one or more editors, use the \hysguardeditor
%% command below, uncommenting it and writing the Editor(s) names, formatted like this:
%% A.B. Normal, J. Doe, A.N. Author, A.N.Editor
\hysguardeditor{Mauro \textsc{Mucedda}, Federica \textsc{Roscioni}, Damiano G. \textsc{Preatoni}}

%% typeset the guard page, if \hysguardtitle or \hysguardeditor above were set, they
%% wull be used here
\hysguardpage % page I
\clearpage
\hyscolophon  % page II
\clearpage

%% if this is a standard issue and no other front matter is needed, you can ignore
%% what follows, leaving it as is. If not, read on...
\hysfrontpage



%% do not change the following commands
\chapterstyle{komalike} % change page style to standard
\setsecnumdepth{none}   % no chapter numbering
\pagestyle{ruled}       % use the same page style used in articles
\pagenumbering{Roman}   % uppercase roman page numbering

%% start here inserting any other extra pages (e.g prefaces, news, communications...)
%% use standard LaTeX sectioning and commands.

%\chapter{Introduzione}
%\lipsum


\includepdf{Pettersson_Elektronik_Sponsorship_0.pdf}


%% always end with \clear doublepage, so that first article always begin on an odd page
\cleardoublepage
\end{document}
%%% Local Variables: 
%%% mode: latex
%%% TeX-master: t
%%% End: 
