% Abstract file structure example : 
% \abstitle{title here}
% \absauthors{names and superscripts for affiliations here}
% \absaddress{affiliations, starting each one with its superscripts, separate affiliations with a \break}
% \abstext{
% \index{author abbreviated name, to be placed in authors' index}
% \index{create an index entry for each author}
%  The abstract text
% }

%% Abstract title
\abstitle{Cranial size has increased over 133 years in a common bat, \emph{Pipistrellus kuhlii}: a response to changing climate or urbanization?}

%% Author names
\absauthors{A. \textsc{Tomassini}$^1$, P. \textsc{Colangelo}$^1$, P. \textsc{Agnelli}$^2$, G. \textsc{Jones}$^3$, D. \textsc{Russo}$^{3,4}$}

\absaddress{$^1$Dipartimento di Biologia e Biotecnologie ``Charles Darwin'', Università degli Studi di Roma ``La Sapienza'', Roma, Italy\break
$^2$Museo di Storia Naturale dell’Università di Firenze, Sezione di Zoologia ``La Specola'', Florence, Italy\break
$^3$School of Biological Sciences, University of Bristol, Bristol, UK\break
$^4$Wildlife Research Unit, Laboratorio di Ecologia Applicata, Dipartimento di Agraria, Università degli Studi di Napoli Federico II, Portici, Italy}

%% Abstract text
\abstext{
%% Author names for index. State each author separately using \index{Doe J.}
\index{Tomassini A.}
\index{Colangelo P}
\index{Agnelli P.}
\index{Jones G.}
\index{Russo D.}
%% The actual abstract text goes here
Bats are promising candidates for studying morphometric responses to anthropogenic climate or land-use changes. We assessed whether the cranial size of a common bat (\emph{Pipistrellus kuhlii}) had changed between 1875 and 2007. We formulated the following hypotheses: (1) if heat loss is an important reaction to climate warming, body size will have decreased in response to the increased temperatures, because small bats have a larger surface-to-volume ratio and dissipate heat more effectively; (2) if water loss is the main driver, body size will have increased in response to the temperature increase, because larger bats will lose water more slowly through a reduced surface-to-volume ratio; (3) the energetic benefits provided by urbanization (food concentration at street lamps, warmer maternity roosts in buildings) will lead to a general body size increase in \emph{P. kuhlii}; and (4) because street lamps impair moth antipredatory manoeuvres, cranial size may have selectively increased as an adaptive response to handle larger prey (moths) in artificially illuminated sites. Ours is the first study to assess temporal trends in bat body size and climate or urbanization over more than a century as possible causal factors. Our study was set in mainland Italy. We used traditional morphometrics to compare seven variables of skull size in 117 museum specimens (75 female, 42 male). Cranial size increased after 1950, but this change was not paralleled by an increase in body size, measured as forearm length. This selective increase matched a rapid increase in electric public illumination in Italy. Street lights are crucial foraging sites for \emph{P. kuhlii}. The directional change that we found in cranial size might represent a microevolutionary adaptive tracking of a sudden shift in food size, making more profitable prey available.
} %% remember to close the abstract text block brace!!