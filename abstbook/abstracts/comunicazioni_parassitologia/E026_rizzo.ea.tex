% Abstract file structure example : 
% \abstitle{title here}
% \absauthors{names and superscripts for affiliations here}
% \absaddress{affiliations, starting each one with its superscripts, separate affiliations with a \break}
% \abstext{
% \index{author abbreviated name, to be placed in authors' index}
% \index{create an index entry for each author}
%  The abstract text
% }

%% Abstract title
\abstitle{Individuazione di Alphacoronavirus nella chirotterofauna nord-occidentale: risultati preliminari}

%% Author names
\absauthors{F. \textsc{Rizzo}$^1$, S. \textsc{Robetto}$^3$ , C. \textsc{Guidetti}$^3$, C. \textsc{Lo Vecchio}$^1$, S. \textsc{Zoppi}$^1$, A. \textsc{Dondo}$^1$, M. \textsc{Ballardini}$^1$, W. \textsc{Mignone}$^1$, L. \textsc{Bertolotti}$^2$, S. \textsc{Rosati}$^2$, M. \textsc{Calvini}$^4$, R. \textsc{Toffoli}$^4$, R. \textsc{Orusa}$^3$, M.L. \textsc{Mandola}$^1$}

\absaddress{$^1$Istituto Zooprofilattico del Piemonte, Liguria e Valle d'Aosta (IZS PLV), Turin, Italy\break
$^2$Department of Veterinary Science, University of Turin, Italy\break
$^3$National Reference Centre for Diseases of Wild Animals (Ce.RMAS), IZS PLV, Aoste, Italy\break
$^4$Chirosphera, Associazione per lo studio e la tutela dei Chirotteri e l’ambiente, Italy}

%% Abstract text
\abstext{
%% Author names for index. State each author separately using \index{Doe J.}
\index{Rizzo F.}
\index{Robetto S.}
\index{Guidetti C.}
\index{Lo Vecchio C.}
\index{Zoppi S.}
\index{Dondo A.}
\index{Ballardini M.}
\index{Mignone W.}
\index{Bertolotti L.}
\index{Rosati S.}
\index{Calvini M.}
\index{Toffoli R.}
\index{Orusa R.}
\index{Mandola M.L.}
%% The actual abstract text goes here
Negli ultimi due decenni alcune specie di chirotteri, a lungo ritenute coinvolte nella sola trasmissione dei virus rabidi, vengono sempre più associate, in veste di \textit{reservoir}, ad agenti zoonosici emergenti, tra i quali SARS-CoV, MERS-CoV NR, HEV e il virus Ebola, destando attenzione crescente a livello mondiale per gli eventuali risvolti sulla sanità pubblica. I pipistrelli, infatti, hanno la capacità di ospitare virus geneticamente molto diversi tra loro senza manifestare segni clinici di malattia. La maggior parte dei \emph{Coronavirus} (CoV) normalmente infetta solo una specie animale o, al massimo, un piccolo numero di specie strettamente correlate. Tuttavia alcuni CoV, come quello responsabile della SARS, emerso per la prima volta nel 2002 in Cina e il MERS-CoV, isolato per la prima volta nell’uomo nel 2012 in Arabia Saudita, hanno superato la barriera di specie infettando ospiti inusuali.

Questo studio nasce con lo scopo di sviluppare un progetto di sorveglianza attiva e passiva sulla circolazione di CoV nella popolazione di chirotteri nel territorio di competenza dell’IZS PLV. 

Da luglio 2013 a oggi 111 esemplari, appartenenti a 16 differenti specie, sono stati catturati con l’ausilio di mist-net e harp-trap nel corso di 19 sessioni di cattura condotte da chirotterologi autorizzati dal Ministero dell’Ambiente (sorveglianza attiva). Parametri biometrici e fisiologici tra i quali, sesso, lunghezza dell’avambraccio, peso, stato riproduttivo sono stati collezionati da ogni esemplare. Prima del rilascio sono stati raccolti, da ogni soggetto, tamponi boccali, rettali e campioni di urina; sono stati, inoltre, conferiti esemplari rinvenuti morti o morti successivamente al ricovero nei Centri di Recupero per gli Animali Selvatici, per un totale di 26 soggetti (sorveglianza passiva). Per escludere il rischio di esposizione a virus rabidi, tutte le carcasse sono state preliminarmente sottoposte a necroscopia, quindi a immunofluorescenza diretta in laboratorio di sicurezza BSL3, con esito negativo.

Su 26 carcasse, 37 tamponi boccali, 15 tamponi rettali e 15 campioni di urina sono state condotte analisi biomolecolari per la ricerca di CoV mediante One step RT-PCR, costruita sulla regione target conservata RdRp. Cinque esemplari, appartenenti alle specie \emph{Myotis nattereri} (n=2), \emph{Rhinolophus ferrumequinum} (n=1) e \emph{Pipistrellus kuhlii} (n=2), collezionati in tre siti diversi del Piemonte, hanno dato esito positivo per la presenza del virus. Le sequenze virali, ottenute a partire dai tamponi rettali e da urina appartenenti ai due esemplari di \emph{M. nattereri}, sono stati caratterizzati filogeneticamente, mostrando un’alta omologia di sequenza (94.55\%) con un ceppo alphaCoV trovato in un esemplare di \emph{M. nattereri} ungherese. Analisi di approfondimento sono in corso sui restanti 3 esemplari. In merito alla presenza dei CoV nei chirotteri, è stato osservato un legame evolutivo tra alcuni bat-CoV e i loro ospiti. È stato riportato che molti CoV dei pipistrelli sembrano essere specie-specifici. In molti casi i bat-CoV sembrano essere associati più strettamente alla specie d’ospite che all’area di campionamento; infatti, la stessa specie può albergare lo stesso tipo virale anche in aree geografiche molto distanti e, allo stesso tempo, diverse specie di pipistrello che condividono lo stesso ambiente possono trasportare CoV differenti. Questo potrebbe spiegare la ragione per la quale l’alphaCoV trovato in \emph{M. nattereri} nel nostro studio sia paragonabile a un ceppo trovato nella stessa specie in Ungheria invece che a un alphaCoV trovato recentemente in un \emph{M. blythii} in Italia nordorientale.
} %% remember to close the abstract text block brace!!