% Abstract file structure example : 
% \abstitle{title here}
% \absauthors{names and superscripts for affiliations here}
% \absaddress{affiliations, starting each one with its superscripts, separate affiliations with a \break}
% \abstext{
% \index{author abbreviated name, to be placed in authors' index}
% \index{create an index entry for each author}
%  The abstract text
% }

%% Abstract title
\abstitle{Il ruolo dei Chirotteri nell'ecologia dei coronavirus emergenti}

%% Author names
\absauthors{M.A. \textsc{De Marco}$^1$, M.R. \textsc{Castrucci}$^2$}

\absaddress{$^1$Istituto Superiore per la Protezione e la Ricerca Ambientale, Via Ca' Fornacetta 9, Ozzano Emilia (BO), Italy\break
$^2$Dipartimento di Malattie Infettive, Parassitarie e Immuno-Mediate, Istituto Superiore di Sanità, Viale Regina Elena 299, Roma, Italy}

%% Abstract text
\abstext{
%% Author names for index. State each author separately using \index{Doe J.}
\index{De Marco A.M.}
\index{Castrucci M.R.}
%% The actual abstract text goes here
Oltre il 60\% delle malattie trasmissibili emergenti nella popolazione umana è di origine zoonotica e circa il 70\% di queste deriva dalla fauna selvatica. L’assenza di una linea di demarcazione tra la medicina umana e quella veterinaria è alla base di un approccio ecologico alla salute dell’uomo e degli animali, comunemente definito ``\textit{One medicine}/\textit{One health approach}''.

Attraverso criteri multidisciplinari di valutazione di una ``salute globale'  umana/animale/ambientale, saranno illustrati i fattori in grado di determinare e modulare lo ``\textit{spillover}'' di alcuni patogeni emergenti dal serbatoio animale. Più in dettaglio, sarà analizzato il potenziale ruolo dei Chirotteri nei circuiti di trasmissione interspecie di alcuni patogeni causa di malattie nella popolazione umana, ed in particolare sarà discusso il ruolo epidemiologico di tali mammiferi nell’ecologia dei coronavirus (CoV).

I CoV, causa di infezioni respiratorie ed enteriche nell’uomo e negli animali, sono talvolta all’origine di importanti epidemie nella popolazione umana, associate a CoV emergenti da un serbatoio animale. Esempi recenti sono la trasmissione zoonotica del CoV della SARS (Severe Acute Respiratory Syndrome) che nel 2003 ha causato una grave epidemia a livello mondiale (oltre 8000 casi umani) e l’emergenza nell’aprile 2012 di un nuovo CoV responsabile della MERS (Middle East Respiratory Syndrome), grave malattia respiratoria tuttora presente e che ad oggi (02-09-2015, World Health Organization) ha causato 1493 casi umani. Anche se altre specie animali, come la Civetta delle palme mascherata (\emph{Paguma larvata}) per il SARS-CoV e il Dromedario (\emph{Camelus dromedarius}) nel caso del MERS-CoV, hanno costituito la principale fonte di contagio per l’uomo, entrambi i virus hanno avuto origine da un serbatoio animale probabilmente rappresentato dai pipistrelli.

Nel novembre 2014 ha avuto inizio il progetto di ricerca ``Emerging respiratory viruses: monitoring of coronavirus infections at the human-animal interface'' del Ministero della Salute, coordinato dall’Istituto Superiore di Sanità (ISS) in collaborazione con Public Health England (PHE), Istituto Superiore per la Protezione e la Ricerca Ambientale (ISPRA), Istituto Zooprofilattico Sperimentale della Lombardia e dell’Emilia-Romagna (IZSLER).

Nell’ambito di tale progetto di ricerca sull’ecologia dei CoV e sul rischio di infezione umana, l’ISPRA ha il compito di pianificare e coordinare la raccolta di campioni di siero da categorie di individui esposti a pipistrelli, possibile serbatoio naturale di CoV emergenti. Le attività previste includono: i) l’arruolamento, su base volontaria, di individui a contatto diretto con pipistrelli durante attività occupazionali e/o ricreazionali (come chirotterologi e persone operanti in centri di recupero fauna selvatica) o esposti indirettamente a livello ambientale (come speleologi); ii) la raccolta dei campioni ematici dalle suddette categorie di individui esposti a chirotteri, da parte di personale sanitario abilitato. I sieri raccolti saranno successivamente esaminati presso l’ISS e/o PHE (UK) per valutare, attraverso la ricerca di anticorpi specifici, l’eventuale esposizione a CoV circolanti nelle popolazioni di pipistrelli.
Attraverso un approccio ecologico, volto alla conservazione delle specie e tutela della salute umana, i risultati ottenuti nel progetto contribuiranno a fornire ulteriori indicazioni sulla dinamica delle malattie infettive emergenti.
} %% remember to close the abstract text block brace!!