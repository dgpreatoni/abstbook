% Abstract file structure example : 
% \abstitle{title here}
% \absauthors{names and superscripts for affiliations here}
% \absaddress{affiliations, starting each one with its superscripts, separate affiliations with a \break}
% \abstext{
% \index{author abbreviated name, to be placed in authors' index}
% \index{create an index entry for each author}
%  The abstract text
% }

%% Abstract title
\abstitle{The relevance of White Nose Syndrome in Europe}

%% Author names
\absauthors{S. \textsc{Leopardi}$^{1,2}$, D. \textsc{Blake}$^2$, S. \textsc{Puechmaille}$^{3,4}$}

\absaddress{$^1$FAO and National Reference Centre for Rabies, OIE Collaborating Centre for Diseases at the Animal/Human Interface, Istituto Zooprofilattico Sperimentale delle Venezie, Viale dell’Università 10, 35020 Legnaro, Italy\break
$^2$Pathology and Pathogen Biology, Royal Veterinary College, London NW1 0TU, UK\break
$^3$Zoology Institute, Ernst-Moritz-Arndt University, Greifswald D – 17489, Germany\break
$^4$School of Biology \& Environmental Science, University College Dublin, Dublin 4, Ireland}

%% Abstract text
\abstext{
%% Author names for index. State each author separately using \index{Doe J.}
\index{Leopardi S.}
\index{Blake D.}
\index{Puechmaille S.}
%% The actual abstract text goes here
Named after the typical white fungal growth around the muzzle, the White Nose Syndrome (WNS) is an emerging disease caused by the cold adapted fungus \emph{Pseudogymnoascus destructans (Pd)} and affecting hibernating bats. The first evidence dates back to the winter 2006--2007, when unprecedented numbers of deaths were reported in mines and caves around New York. Since then, WNS has rapidly spread across the eastern USA and Canada killing an estimated six million bats; at the end of this hibernating season, it was confirmed in seven bat species from 27 US states and five Canadian provinces. 

The lack of reports of the disease and its causative agent prior 2006, its rapid spread and the high mortality rate are all suggestive of the pathogen being novel in the area. The presence of \emph{Pd} in Europe was first recorded in France in 2009 and has now been confirmed throughout the temperate regions of the continent. Historical photographic records also suggest that it was probably present well before 2006, when it emerged as a deadly disease in New York. However, no winter mass mortality associated with the disease has been reported to date in Europe, listing it as a likely source for the recent introduction of the pathogen to North America. 

This hypothesis was tested by analyzing American and European isolates of \emph{Pd} on a molecular level. Twenty-eight European isolates from across Europe were analyzed using a previously published multi-locus sequence typing (MLST) panel and aligned with equivalent American sequences. While in North America a single clone was found over time and across the range of expansion of the pathogen, eight different haplotypes have been confirmed in Europe. The higher degree of genetic diversity in the latter population, paired with its benign outcome in bats is consistent with a longer presence of \emph{Pd} in Europe and strongly supports a recent introduction in North American. Furthermore, 100\% similarity was found between sequences from American \emph{Pd} and the most widespread European haplotype, providing evidence that Europe is the likely source population. 

This study provides for the first time strong evidence for a European origin of the fungus, confirming an already widespread speculation. Given that there is no bat migrating between North America and Europe, it is very likely that the fungus has been introduced via anthropogenic activities. These results should increase the awareness about the implications of human behavior and activities in triggering emergence of disease in wildlife other than in the human population itself. Red squirrel pox, varroasis and chytridiomycosis are other examples of diseases emerged due to the introduction of pathogens into naïve populations as the undesired consequence of the unprecedented movement of humans, animals and other media such as agriculture materials or ballast water on a global scale. Control measures recommended to reduce the risk for pathogen pollution are the adoption of disease risk analysis during animal transport and an higher level of biosecurity during animal sampling. 

Even if the risk associated with white nose syndrome is currently considered to be low for European bats, no experimental studies have been performed in order to prove the hypothesis of an innate resistance or a behavioral adaption related to co-evolution with the pathogen, so that the emergence of the clinical disease cannot be completely excluded. In this perspective continuous surveillance and a strict monitoring of the affected populations should be implemented in Europe for an early detection of mortality events. 

As WNS continues to spread and more bat populations are being devastated in North America every winter it is more and more clear how the management of the disease doesn’t guarantee recovery of severely affected species without being associated with specific conservation plans. Unlike other small mammals, bats are long living and have low reproductive output, which prevents these animals to recover quickly from population decline. Furthermore some species are forming large colonies, and the size of viable populations have not been investigated. The hope is that the lesson of WNS will be learned on a global scale by increasing efforts in ensuring a good quality habitat for bat populations and a mitigation of all other anthropogenic sources of mortality.
} %% remember to close the abstract text block brace!!