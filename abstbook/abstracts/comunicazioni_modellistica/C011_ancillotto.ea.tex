% Abstract file structure example : 
% \abstitle{title here}
% \absauthors{names and superscripts for affiliations here}
% \absaddress{affiliations, starting each one with its superscripts, separate affiliations with a \break}
% \abstext{
% \index{author abbreviated name, to be placed in authors' index}
% \index{create an index entry for each author}
%  The abstract text
% }

%% Abstract title
\abstitle{A winning bat in a changing environment: local and global responses of \emph{Pipistrellus kuhlii} to urbanization and climate change}

%% Author names
\absauthors{L. \textsc{Ancillotto}$^1$,  L. \textsc{Santini}$^2$, N. \textsc{Ranc}$^3$, L. \textsc{Maiorano}$^2$,  A. \textsc{Tomassini}$^4$, D. \textsc{Russo}$^{1,5}$}

\absaddress{$^1$Wildlife Research Unit, Laboratorio di Ecologia Applicata, Sezione di Biologia e Protezione dei Sistemi Agrari e Forestali, Dipartimento di Agraria, Università degli Studi di Napoli Federico II, via Università 100, 80055 Portici (Napoli), Italy\break
$^2$Dipartimento di Biologia e Biotecnologie ``Charles Darwin'', Università degli Studi di Roma La Sapienza\break
$^3$Organismic and Evolutionary Biology Department, Harvard University, 26 Oxford Street, Cambridge, MA 02138, United States\break
$^4$Tutela Pipistrelli – ONLUS, via Lodovico Bertonio, 20 – 00126 Roma\break
$^5$School of Biological Sciences, University of Bristol, Bristol, UK}

%% Abstract text
\abstext{
%% Author names for index. State each author separately using \index{Doe J.}
\index{Ancillotto L.}
\index{Santini L.}
\index{Ranc N.}
\index{Maiorano L.}
\index{Tomassini A.}
\index{Russo D.}
%% The actual abstract text goes here
Urbanization and climate change are two major global threats to biodiversity, including bats, both leading to biological homogenization, i.e. the process of a few adaptable species spreading while more sensitive ones decrease and eventually go extinct. 
We selected Kuhl’s pipistrelle \emph{Pipistrellus kuhlii}, a bat species typical of Mediterranean habitats and often associated with urban settlements, to test for the effects of urbanization and climate change on its local ecology and global distribution, predicting that a synurbic and Mediterranean species such as \emph{P. kuhlii} will benefit from both climate change and urbanization. 

We observed small-scale effects of urbanization, focusing on land use change and light pollution, by measuring individual fitness, e.g. female productivity and birth timing, in reproductive roosts of peninsular Italy. We analysed the European distribution of \emph{P. kuhlii} and its recent range expansion by applying species distribution models for assessing the roles of climate change and urbanization (or the synergic action of the two factors) as drivers of \emph{P. kuhlii} recent range dynamics. 

We found evidence of a positive effect of urbanization, in terms of both land cover and light pollution, on individual fitness at a local scale: females produced more pups in roosts surrounded by discontinuous urban land cover, particularly where light pollution levels were higher. At a global scale, only climate though seems to have a relevant role in driving the range expansion of \emph{P. kuhlii} in Europe in the last decades. 

From our analyses, in a rapidly changing environment, \emph{P. kuhlii} is definitely a winner among bats, being able to cope and even take advantage of both local and global-scale human induced environmental modifications, thus having the potential to impact on the bat communities in the near future. 
} %% remember to close the abstract text block brace!!