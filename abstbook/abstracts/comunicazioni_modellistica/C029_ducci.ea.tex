% Abstract file structure example : 
% \abstitle{title here}
% \absauthors{names and superscripts for affiliations here}
% \absaddress{affiliations, starting each one with its superscripts, separate affiliations with a \break}
% \abstext{
% \index{author abbreviated name, to be placed in authors' index}
% \index{create an index entry for each author}
%  The abstract text
% }

%% Abstract title
\abstitle{Testing the effectiveness of protected areas to preserve bat habitat and commuting routes: a region-scale model for the aerial-hawker \emph{Nyctalus noctula}}

%% Author names
\absauthors{L. \textsc{Ducci}$^{1,2}$, P. \textsc{Agnelli}$^2$, M. \textsc{Di Febbraro}$^3$, L. \textsc{Frate}$^{3,4}$, D. \textsc{Russo}$^{5,6}$, M.L. \textsc{Carranza}$^3$, A. \textsc{Loy}$^3$, G. \textsc{Santini}$^1$, F. \textsc{Roscioni}$^3$}

\absaddress{$^1$Dipartimento di Biologia, Università degli Studi di Firenze, Firenze, Italy\break
$^2$Museo di Storia Naturale, Università degli Studi di Firenze, Firenze, Italy\break
$^3$EnvixLab, Dipartimento di Bioscienze e Territorio, Università degli Studi del Molise, Contrada Fonte Lappone snc, I-86090 Pesche, Italy\break
$^4$Istituto di Biologia Agro-Ambientale e Forestale, CNR/IBAF, Monterotondo, Roma\break
$^5$Wildlife Research Unit, Laboratorio di Ecologia Applicata, Dipartimento di Agraria, Università degli Studi di Napoli Federico II, Portici (NA), Italy\break
$^6$School of Biological Sciences, University of Bristol, Woodland Road BS8 1UG, Bristol, U.K.}

%% Abstract text
\abstext{
%% Author names for index. State each author separately using \index{Doe J.}
\index{Ducci L.}
\index{Agnelli P.}
\index{Di Febbraro M.}
\index{Frate L.}
\index{Russo D.}
\index{Carranza M.L.}
\index{Loy A.}
\index{Santini G.}
\index{Roscioni F.}
%% The actual abstract text goes here
Habitat fragmentation is a key driver of biodiversity loss since it decreases dispersal success, increases mortality and reduces genetic diversity. Re-establishing links among formerly connected natural habitats is thus imperative to maintain biological diversity. We propose a method based on Species Distribution Models (SDMs) and functional connectivity analysis for a highly mobile bat species, \emph{Nyctalus noctula}, to evaluate the effectiveness of protected areas (PAs) in preserving suitable habitats and potential commuting corridors (PCCs) for bats. In the study, set in central Italy (Tuscany), SDMs were trained with bat presence records and elevation, hydrographic network and land cover as environmental predictors. The SDM output and a set of environmental proxies of commuting routes were used to build a resistance layer for the connectivity analysis. Resulting PCCs were ranked according to their relevance. The effectiveness of PAs was assessed by overlapping the PAs map with the SDM and the PCC outputs. We identified several critical areas requiring protection regimes in order to preserve suitable habitat and functional connectivity for the species. Our results highlighted that PAs cover just a small portion of \emph{N. noctula}’s suitable habitat (20\%), although they include sites scoring the highest probability of presence. Moreover, only a small part (20\%) of the less relevant PCCs is included in the regional PAs. The method we propose proved efficient to identify species-specific critical areas that deserve immediate conservation actions. In addition, because of its flexibility, our approach can be easily extended to other taxa and in other geographical and conservation contexts.
} %% remember to close the abstract text block brace!!