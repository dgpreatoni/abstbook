% Abstract file structure example : 
% \abstitle{title here}
% \absauthors{names and superscripts for affiliations here}
% \absaddress{affiliations, starting each one with its superscripts, separate affiliations with a \break}
% \abstext{
% \index{author abbreviated name, to be placed in authors' index}
% \index{create an index entry for each author}
%  The abstract text
% }

%% Abstract title
\abstitle{Evaluating the impact of climate change on species distribution. A study on the Mediterranean bat community of western Palaearctic}

%% Author names
\absauthors{M. \textsc{Piccioli Cappelli}$^1$, A. \textsc{Martinoli}$^1$, D. \textsc{Russo}$^{2,3}$, H. \textsc{Rebelo}$^4$}

\absaddress{$^1$Unità di Analisi e Gestione delle Risorse Naturali – \textit{Guido Tosi Research Group}, Dipartimento di Scienze Teoriche e Applicate, Università degli Studi dell’Insubria, Via J. H. Dunant 3, I - 21100 Varese, Italy\break
$^2$Wildlife Research Unit, Laboratorio di Ecologia Applicata, Dipartimento di Agraria, Università degli Studi di Napoli Federico II, via Università 100, 80055 Portici, Napoli, Italy\break
$^3$School of Biological Sciences, University of Bristol, Bristol, United Kingdom\break
$^4$CIBIO/InBIO, R. Padre Armando Quintas, 4485-661 Vairão, Portugal}

%% Abstract text
\abstext{
%% Author names for index. State each author separately using \index{Doe J.}
\index{Piccioli Cappelli M.}
\index{Martinoli A.}
\index{Russo D.}
\index{Rebelo H.}
%% The actual abstract text goes here
Biodiversity loss has been increasing since the second half of the 20\textsuperscript{th} century, and is likely to continue into the future. Ecosystems and their services are expected to suffer additional stresses due to temperature increases and changes in precipitation patterns driven by current climate change. For the 21\textsuperscript{st} century, climate models forecast increasing drought over different regions of the globe. Particularly, the Mediterranean Basin is predicted to suffer heavy drought episodes and an overall dryness trend during the next decades.

The main goal of this work is to study how ongoing climate change will influence the distribution of Mediterranean and temperate bats (Mammalia, Chiroptera) in the Western Palaearctic, with a particular focus on future dry areas. Using species distribution modelling techniques (maximum entropy algorithm, MaxEnt) and a set of 7 bioclimatic variables, we predicted the current potential climatic niches of 24 bat species, and then projected them for the years 2061--2080 using three global circulation models (CCSM4, HadGEM2-ES and MIROC-ESM) under three climate scenarios (RCP2.6, RCP6.0 and RCP8.5). First, for each species we assessed niche changes (gain, loss and persistence areas) between present and future conditions. Second, individual models were averaged within each RCP. For each future scenario, resulting ensemble forecasts were used to calculate the direction and the magnitude of niche displacements and to predict variations in species richness and community turnover. Stable areas where most  species are expected to preserve their current niche (``warm'' refugia) were also highlighted. Last, we determined the extinction risk for every species analysing the variation in the occupied area and the overlap between the estimated current and future niches.

Assuming unlimited dispersal, we observed an overall north-northwest shift of the bat community. Consequentially, by the 2070s we predicted an increase in species number in Central Europe and a decrease in the southern part of the study area. Depending on the climate scenario considered, mean range centroid potentially shifted from 394 km (RCP2.6) to 719 km (RCP8.5) in a direction between 349.2\degree{} (RCP8.5) and 355.1\degree{} (RCP6.0). The average decadal rate of displacement ranged from 41.5 km (RCP2.6) to 75.6 km (RCP8.5). More than 50\% of the studied species are projected to undergo increases in their climatic niche size and a good connectivity between the current and the future distribution also under the most dramatic scenario (RCP 8.5), whilst 17\% of the species could face an increased extinction risk already for the less dramatic scenario (RCP 2.6) due to a strong range contraction and scarce overlap. Several areas could act as \textit{in situ} macrorefugia and may represent important corridors between the current and future ranges as well as potential areas from which species might be able to expand if climatic conditions become favourable again. In general, accordingly to the intermediate scenario (RCP 6.0) we found that the regions where most (>70\%) of the modelled species would preserve their current niche under climate change will be approximately northern and eastern Iberia, portions of France (western France, northern Massif Central, Provence and the Maritime Alps), the Apennines, High and Upper Rhine, the northern part of Turkey (Black Sea region) and isolated patches in southern England, northern Algeria and northern Tunisia.

In agreement with other studies, our results show a poleward shift of the  bat community and a loss of diversity in the lower latitudes and altitudes as a consequence of climate change. Albeit many species seem to be possibly favoured by climate change, the herein predicted changes in size and geographical position of the climatic niches suggest a potential high extinction risk for endemic and rare species, thus representing priority targets for research and conservation. Considering the importance of the  ``warm'' refugia we identified for both their currently high species richness and their stability over time, these areas should be primary candidates for immediate protection in order to maximize species’ persistence over time and to maintain landscape connectivity to the potential future suitable climatic areas (\textit{ex situ} refugia). Therefore, these areas should be managed to increase their resilience and long-term sustainability required to maintain biodiversity and ecosystem services. Despite our bioclimatic envelopes do not consider important limiting variables still difficult to model (e.g. species interactions, phenotypic plasticity, diseases, land-use changes), they provide an overview of the impact of current climate change on bat distribution over a large-scale area and could represent useful tools for land-managers. Further studies may evaluate the dispersal ability of most vulnerable species and the impact of biogeographic barriers, namely if expanding species would be capable to reach future suitable areas. 
} %% remember to close the abstract text block brace!!