% Abstract file structure example : 
% \abstitle{title here}
% \absauthors{names and superscripts for affiliations here}
% \absaddress{affiliations, starting each one with its superscripts, separate affiliations with a \break}
% \abstext{
% \index{author abbreviated name, to be placed in authors' index}
% \index{create an index entry for each author}
%  The abstract text
% }

%% Abstract title
\abstitle{A theory on the evolution of \emph{Rhinolophus ferrumequinum} and \emph{Myotis blythii} in the Recent Italian Quaternary}

%% Author names
\absauthors{A. \textsc{Pereswiet-Soltan}}

\absaddress{Institute of Systematics and Evolution of Animals, Polish Academy of Sciences and  
Club Speleologico Proteo, Vicenza, Italy}

%% Abstract text
\abstext{
%% Author names for index. State each author separately using \index{Doe J.}
\index{Pereswiet-Soltan A.}
%% The actual abstract text goes here
In recent years phylogeographic studies on bats have been principally based on genetic analyses and barely considered the evolution of the skeletal apparatus which, in particular the oral one, can be compared with fossil remains. This study reports the morphometric analysis of hemimandible and teeth of fossil remains of bats found during archaeological excavations, which were carried out in some Italian caves in order to deepen both general knowledge and history of bats evolution in the Italian Late Pleistocene and the Early Holocene

The study is focused on two highly specialized species, for which collected findings were sufficiently complete and abundant: \emph{Rhinolophus ferrumequinum} (Schreber, 1774) and \emph{Myotis blythii} (Tomes, 1857). For to the first species, a comparison was carried out between the specimen from the layers I-O (dated 40000--45000 years BP, Greenland Stadial 9--12) in the cave ``Grotta del Broion'  (North-Eastern Italy, territory of Vicenza) and the specimen from Northern Italy and Algeria. For the second one, the comparison with the recent specimen from Northern Italy and Algeria affected the layers I-O and P-Q (45000 years BP, Heinrich event 5) in the cave ``Grotta del Broion'' and those from layer 8 g to 4 (about 14000--9000 years BP, from Dryas II to Preboreal) in the cave ``Grotta della Serratura'' (South-Western Italy, territory of Salerno). For this analysis 17 measurements, carried out on teeth or hemimandible, were considered. The comparative analysis of the measurements on fossil and current material reveals a remarkable dimensional difference between the fossil specimen of \emph{Rhinolophus ferrumequinum} from the cave ``Grotta del Broion'', which have a greater chewing apparatus, and the current ones, among others the Algerian ones, which are smaller. The analysis of \emph{Myotis blythii} is more problematic, as at present there are three ``sister'' species, in order of size: \emph{Myotis myotis} (in Italy and continental Europe), \emph{Myotis punicus} (in Northern-Africa, Sardinia, Malta and Crete) and \emph{Myotis blythii} (in the Mediterranean-European region). Moreover, the differences found for \emph{Myotis blythii} are not as evident as the ones found in \emph{Rhinolophus ferrumequinum}. The results show that the findings from ``Grotta del Broion'' are not different from the current ones from Northern-Italy, while the ones  from ``Grotta della Serratura'' can be dimensionally classified between \emph{Myotis punicus} and \emph{Myotis blythii}, becoming the size of the most recent specimens smaller and smaller.

This leads to the formulation of two hypotheses, which need a future analysis: 1) when the sea level was lower in the Early Pleistocene in Southern Italy, \emph{Myotis punicus} lived, which afterwards whether evolved or was supplanted by \emph{Myotis blythii}; 2) the evolution of \emph{Myotis blythii} has taken place with the reduction of its chewing apparatus. The comparative analysis of the two species leads to the formulation of two further hypotheses: 1) the size change follows Bergmann's Rule, with greater size bats of the same species present in the coldest periods; 2) \emph{Myotis blythii} has an older evolutionary history, as its oldest findings are dated back to the Middle Pliocene, so that the differences between fossil and current material are less evident than in the case of \emph{Rhinolophus ferrumequinum}, which findings are dated back to the Middle Pleistocene and is still evolving.
} %% remember to close the abstract text block brace!!