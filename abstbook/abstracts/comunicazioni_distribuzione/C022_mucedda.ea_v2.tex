% Abstract file structure example : 
% \abstitle{title here}
% \absauthors{names and superscripts for affiliations here}
% \absaddress{affiliations, starting each one with its superscripts, separate affiliations with a \break}
% \abstext{
% \index{author abbreviated name, to be placed in authors' index}
% \index{create an index entry for each author}
%  The abstract text
% }

%% Abstract title
\abstitle{Note sui pipistrelli nelle piccole isole della Sardegna}

%% Author names
\absauthors{M. \textsc{Mucedda}, E. \textsc{Pidinchedda }, M. L. \textsc{Bertelli}}

\absaddress{Centro Pipistrelli Sardegna, Via G. Leopardi 1 – 07100 Sassari, Italy. email: \url{batsar@tiscali.it}}

%% Abstract text
\abstext{
%% Author names for index. State each author separately using \index{Doe J.}
\index{Mucedda M.}
\index{Pidinchedda E.}
\index{Bertelli M.L.}
%% The actual abstract text goes here
\textbf{Questo lavoro è presente anche in forma di \textit{extended abstract} a pag. \pageref{ext:E022}}

È stato realizzato uno studio sui chirotteri nelle piccole isole della Sardegna, tendente a stabilire quali specie siano presenti. Le indagini sono state condotte mediante esplorazione di rifugi (edifici, grotte, gallerie sotterranee), monitoraggio con Bat Detector e raramente mediante catture notturne con le reti.

Oggetto della ricerca sono state complessivamente 15 isole, a partire da nord: La Maddalena, Caprera, Santo Stefano, Spargi, Budelli, Santa Maria, Tavolara, Molara, Figarolo, Asinara, Piana, San Pietro, Sant’Antioco, Serpentara e Cavoli.

Sulle isole maggiori La Maddalena, Caprera, Tavolara, Asinara, San Pietro e Sant'Antioco le ricerche sono state più approfondite e protratte negli anni, mentre sulle isole minori le indagini sono state più ridotte, limitate talvolta ad una sola notte di monitoraggio. 

I dati esistenti in bibliografia sono molto limitati, riferiti solamente a pochissime specie per le isole di San Pietro, Sant’Antioco, La Maddalena e Tavolara.

Su 21 specie di chirotteri presenti nella Sardegna, almeno 11 sono state riscontrate nel totale delle isole minori. Si tratta di \emph{Rhinolophus ferrumequinum}, \emph{Rhinolophus hipposideros}, \emph{Miniopterus schreibersii}, \emph{Myotis capaccinii}, \emph{Myotis daubentonii}, \emph{Pipistrellus pipistrellus}, \emph{Pipistrellus kuhlii}, \emph{Pipistrellus pygmaeus}, \emph{Hypsugo savii}, \emph{Tadarida teniotis} e la coppia \emph{Eptesicus serotinus}/\emph{Nyctalus leisleri} considerate come un’unica entità in quanto non è stato possibile distinguerle sulla base delle loro emissioni ultrasonore. Le prime 5 specie sono state individuate per osservazione diretta all’interno di rifugi o cattura, mentre le altre sono state contattate mediante registrazioni bioacustiche. 

Il più alto numero di specie di pipistrelli si riscontra all’Asinara con 10 entità, seguita da Caprera e Tavolara con 8, quindi La Maddalena con 7. Nelle altre isole, soprattutto in quelle più piccole, il numero va a diminuire sino al minimo di una sola specie. Sulla base dei dati presenti in bibliografia, L’Asinara si attesta in cima alle isole italiane con il maggior numero di specie alla pari con l’Isola d’Elba, seguite da Caprera e Tavolara, mentre La Maddalena è alla pari con l’isola del Giglio.

Le specie più ampiamente diffuse sono \emph{Pipistrellus pipistrellus}, presente in tutte le isole, seguita da \emph{Tadarida teniotis} in 12 isole, \emph{Hypsugo savii} e \emph{Pipistrellus kuhlii} in 9 isole. La più rara è risultata invece \emph{Myotis  daubentonii} osservata in una sola isola.
} %% remember to close the abstract text block brace!!