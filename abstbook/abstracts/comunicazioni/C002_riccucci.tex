% Abstract file structure example : 
% \abstitle{title here}
% \absauthors{names and superscripts for affiliations here}
% \absaddress{affiliations, starting each one with its superscripts, separate affiliations with a \break}
% \abstext{
% \index{author abbreviated name, to be placed in authors' index}
% \index{create an index entry for each author}
%  The abstract text
% }

%% Abstract title
\abstitle{Play in Bats: general overview, current knowledge and future challenges}

%% Author names
\absauthors{M. \textsc{Riccucci}}

\absaddress{Gruppo Italiano Ricerca Chirotteri (GIRC) – email: \url{marco.riccucci@gmail.com}}

%% Abstract text
\abstext{
%% Author names for index. State each author separately using \index{Doe J.}
\index{Riccucci M.}
%% The actual abstract text goes here
The study of animal behaviour, and of animal play in particular, did not develop until after the writings of Charles Darwin. In his book \textit{The Descent of Man and Selection in Relation to Sex} (1871), he wrote: ``\textsl{Happiness is never better exhibited than by young animals, such as puppies, kittens, lambs, etc., when playing together, like our own children.}'' Play is useful to the normal process of evolution by natural selection; when animals play, they are practising basic instincts for survival. Although virtually all young mammals play (and adults of many species too), this particular behaviour is still lacking in studies and is quite unknown. Researchers found that play might have immediate benefits and not only delayed ones. Five criteria were defined by Burghardt to characterize a behaviour as play, if all come together. Play in animals is usually classified in three different types: locomotor play, object play and social play, and they can occur at the same time too. Bats are very successful and specialized mammals and can provide many hints and opportunities to study their behaviour even on the evolutionary side, but so far little is known about play in bats. Their nocturnal habits make it difficult to study behaviour in Chiroptera (more than 1300 species in every habitat). Their complex diversity shows many different social and individual behaviours. As reported in Leen (1969) the young of the free-tailed bats, \emph{Tadarida brasiliensis}, ``\textsl{all joint together for the greater part of the day or night to play and tussle, stage sham battles and pursuits, and otherwise romp in a fashion which reminds one of a litter of puppies or kittens}''. Neuweiler (1969) gives an account of mother-offspring play in \emph{Pteropus giganteus}. Play fighting between mother and young seems addressed in the training of the young to adulthood. Some social play has been reported in vampire bats, which show a highly developed social behaviour. Young vampires play together slapping each other with their wings, chasing each other and mutually sniffing. Precursors of echolocation calls of young bats may serve a communication function during the first week prior to its modification and thereafter be used for orientation and navigation, which becomes increasingly important for the survival of young bats. The occurrence of babbling in some species attests to the humanlike development of audio-vocal communication in bats, as found in \emph{Saccopteryx bilineata}. From a neural and functional perspective, babbling may be equivalent to play behaviour (Kanwal et al., 2013). Object play has not been specifically described in bats but a film of several species of fruit bats by the Lubee Foundation provides suggestive evidence for object play using the five criteria. The study of the behaviour of Australian Pteropodidae confirms this view; the juveniles are very active, mutually grooming and playfully fighting and smelling each other. The aggressive behaviour is learnt from the male in the family and later by play in the juvenile group. This behaviour has been confirmed by my observations made in various areas of the world, especially concerning megabats. During my study trip in 1984 to the Seychelles (Indian Ocean), I was able to observe the behaviour of some groups of \emph{Pteropus seichellensis} on Praslin, the second largest island. Many of them live on takamaka tree (\emph{Calophyllum inophyllum}); they do not sleep quietly during the day and are often noisy and squabble with their neighbours and engage themselves in play-fighting, tussle, play-chasing, even adult animals. In 1996 I observed a similar behaviour in \emph{Pteropus giganteus}, one of the largest bats, in Viharamahadevi Park in Colombo, the capital city of Sri Lanka. During my stay in the island of Rodrigues (Indian Ocean) in 2003 I studied several different behaviours of the endemic \emph{Pteropus rodricensis}, already reported especially in captivity: ``play chase'' by immature bats flying to one location and rapidly leaving; ``play wrestle'' involves close belly contact between individuals, with restrained biting on the neck; a pair of bats, or even a group wrestling together, often adult females, rarely adult males; sometimes chase and wrestle alternate in long play sessions. Play in Chiroptera was rated as 1.0 by Iwaniuk et al. (2001); a group of very playful animals, Primates, was rated as 3. Burghardt (2005) raises the ranking of bats to 1.5, based on vampire and fruit bat behaviour. We know little about play in Chiroptera, very little on this behaviour in microbats in particular. It could be helpful to study bats in captivity, even during periods of rehabilitation. In addition, the observation of the behaviour in zoos can give interesting results, although this may chiefly concern the flying foxes, which are easier to raise. However we should be very cautious about drawing any conclusions, as there are many examples of behaviours studied under artificial conditions that, when re-examined by ethologists under natural conditions, have turned out to be distorted.} %% remember to close the abstract text block brace!!