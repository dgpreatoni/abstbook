% Abstract file structure example : 
% \abstitle{title here}
% \absauthors{names and superscripts for affiliations here}
% \absaddress{affiliations, starting each one with its superscripts, separate affiliations with a \break}
% \abstext{
% \index{author abbreviated name, to be placed in authors' index}
% \index{create an index entry for each author}
%  The abstract text
% }

%% Abstract title
\abstitle{Parametri ecologici condizionanti la comunità ectoparassitaria in colonie italiane di Chirotteri}

%% Author names
\absauthors{P. \textsc{Priori}$^1$, L. \textsc{Guidi}$^1$, D. \textsc{Scaravelli}$^2$}

\absaddress{$^1$Dipartimento di Scienze della Terra, della Vita e dell’Ambiente. Università degli Studi di Urbino Carlo Bo, Campus Scientifico, loc. Crocicchia, 61029 Urbino. email: \url{pamela.priori@uniurb.it}\break
$^2$Dipartimento di Scienze Mediche Veterinarie, Università di Bologna, via Tolara di sopra 50, Ozzano Emilia. email: \url{dino.scaravelli@unibo.it}}

%% Abstract text
\abstext{
%% Author names for index. State each author separately using \index{Doe J.}
\index{Priori P.}
\index{Guidi L.}
\index{Scaravelli D.}
%% The actual abstract text goes here
Lo studio delle comunità ectoparassitarie nei Chirotteri presenta numerosi spunti per approfondire le relazioni evolutive  ospite-parassita, essendo le specie ospiti caratterizzate da un’ecologia complessa, con cambi di \textit{roost} stagionali e scelte microclimatiche e di substrato differenziate, oltre che da una fenologia variabile, con una serie di ectoparassiti che si sono adattati a queste particolarità con scelte evolutive davvero inusuali. Per approfondire questi temi stiamo campionando le faunule parassitarie in diverse specie di Chirotteri, correlandole con la specie ospite e la relativa struttura dei \textit{roost}, sia in termini di composizione faunistica, sia di microclima e substrato. In questa analisi abbiamo preso in considerazione i risultati dei campionamenti effettuati su 5 specie di Chirotteri in 6 siti italiani.

I campionamenti avvengono sia a vista, in modo occasionale, sia secondo un protocollo che prevede l’utilizzo di uno spray a base di piretroidi, innocui per i vertebrati e con buon potere abbattente sui parassiti. Questa procedura permette di raccogliere l’intera comunità di ectoparassiti da ogni singolo esemplare ospite. I parassiti di ogni individuo vengono successivamente posti in provette singole con alcool a 70\degree{} per poi essere identificati in laboratorio.

I risultati preliminari qui esposti, su 789 esemplari appartenenti alle specie \emph{Myotis myotis}, \emph{M. blythii}, \emph{M. punicus}, \emph{Miniopterus schreibersii} e \emph{Rhinolophus hipposideros}, rivelano come le colonie campionate siano caratterizzate da una faunula parassitaria nel complesso adeguatamente descritta in letteratura per quanto riguarda la biodiversità, ma ancora da elaborare in termini ecologici e di comunità. La biodiversità della faunula parassitaria è apparsa proporzionale alla numerosità degli ospiti all’interno del rifugio. Colonie con più specie di ospiti presentano un maggior numero di specie di ectoparassiti per ospite e un carico parassitario totale maggiore. Significativa è la differenza tra 2, 3 e 4 specie ospiti presenti nel rifugio, ma non molto significativa tra 4 e 5.

Le specie di Chirotteri che vivono in gruppi più numerosi e insistono su \textit{roost} ipogei umidi e freddi, frequentati da diverse generazioni, mostrano carichi parassitari ingenti in particolare di Nycteribidae (Diptera: Streblidae) e Spinturnicidae (Acarina: Mesostigmata). Non è significativamente diverso il carico tra le tre specie di \emph{Myotis} nelle diverse stazioni di campionamento.

La presenza di Streblidae (Diptera: Hippoboscoidea) si accentra, oltre che in Sardegna, soprattutto in un sito della Toscana centrale. L’unico rappresentante presente nella nostra fauna, \emph{Brachitarsinia flavipennis}, predilige i Rhinolophidae e recentemente è stata ritrovata su \emph{Rhinolophus hipposideros}; tuttavia nelle colonie riproduttive miste si porta a foraggiare anche su \emph{Miniopterus schreibersii} oltre che sui \emph{Myotis} spp.

Gli Spinturnicidae sono molto abbondanti nella coppia \emph{Myotis myotis}/\emph{M. blythii}, in tutti gli ambienti, mentre i rappresentanti delle famiglie Argasidae e Ixodidae (Acarina: Ixodida) sono stati raccolti solo in determinate colonie o risultano numericamente consistenti solo in situazioni sanitarie di crisi, soprattutto nei giovani pipistrelli.

Si sollecita in questo ambito quindi un’ulteriore collaborazione da parte di quanti possano venire a contatto con parassiti per uno studio più ampio e che valuti ulteriori specie e più ambiti ecologici.
} %% remember to close the abstract text block brace!!