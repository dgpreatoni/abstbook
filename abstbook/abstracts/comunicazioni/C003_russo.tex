% Abstract file structure example : 
% \abstitle{title here}
% \absauthors{names and superscripts for affiliations here}
% \absaddress{affiliations, starting each one with its superscripts, separate affiliations with a \break}
% \abstext{
% \index{author abbreviated name, to be placed in authors' index}
% \index{create an index entry for each author}
%  The abstract text
% }

%% Abstract title
\abstitle{Protecting one, protecting both? Scale-dependent ecological differences in two species using dead trees, the \emph{Rosalia} longicorn beetle and the Barbastelle bat}

%% Author names
\absauthors{D. \textsc{Russo}$^{1,2}$, M. \textsc{Di Febbraro}$^3$, L. \textsc{Cistrone}$^4$, G. \textsc{Jones}$^2$, S. \textsc{Smeraldo}$^1$,  A. P. \textsc{Garonna}$^5$,  L. \textsc{Bosso}$^1$}

\absaddress{$^1$Wildlife Research Unit, Dipartimento di Agraria, Università degli Studi di Napoli Federico II, Portici, Napoli, Italy\break
$^2$School of Biological Sciences, University of Bristol, Bristol, United Kingdom\break
$^3$EnvixLab, Dipartimento Bioscienze e Territorio, Università del Molise, Pesche, Italy\break
$^4$Forestry and Conservation, Cassino, Frosinone, Italy\break
$^5$Laboratorio di Entomologia ``Ermenegildo Tremblay'', Dipartimento di Agraria, Università degli Studi di Napoli Federico II, Portici, Napoli, Italy}

%% Abstract text
\abstext{
%% Author names for index. State each author separately using \index{Doe J.}
\index{Russo D.}
\index{Di Febbraro M.}
\index{Cistrone L.}
\index{Jones G.}
\index{Smeraldo S.}
\index{Garonna A. P.}
\index{Bosso L.}
%% The actual abstract text goes here
Organisms sharing the same habitats may differ in small-scale microhabitat requirements or benefit from different management. In this study, set in Italy, we focused on two species of high conservation value, the cerambycid beetle \emph{Rosalia alpina} and the bat \emph{Barbastella barbastellus}, which often share the same forest areas and in several cases the same individual trees. We compared the potential distribution and, at two spatial scales, the niches between such species. The predicted distributions largely overlapped between the beetle and the bat. The niches proved to be similar on a broad scale, yet not on the plot one. Compared with \emph{B. barbastellus},\emph{R. alpina} tends to occur at lower altitude in more irradiated sites with lower canopy closure and uses shorter trees with wider diameters. \emph{B. barbastellus} trees occurred more often within forest or along its edges, whereas \emph{R. alpina} lays eggs in trees found in clearings. \emph{B. barbastellus} plots were more frequent in forest, \emph{R. alpina} ones in forested pasture and open-shredded forest. Overall, exposure to sun influenced more critically site and tree selection by \emph{R. alpina}, as a warm microclimate is essential for larval development. Although \emph{B. barbastellus} reproduction may be favoured by warmer roosting conditions, bats may also find such conditions in dense forest, in strongly-irradiated cavities high up in tall trees that project above the canopy. We emphasize that subtle differences in the ecological requirements of syntopic taxa could be missed at broad scales, so multiple-scale assessment is always advisable.} %% remember to close the abstract text block brace!!