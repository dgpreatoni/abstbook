% Abstract file structure example : 
% \abstitle{title here}
% \absauthors{names and superscripts for affiliations here}
% \absaddress{affiliations, starting each one with its superscripts, separate affiliations with a \break}
% \abstext{
% \index{author abbreviated name, to be placed in authors' index}
% \index{create an index entry for each author}
%  The abstract text
% }

%% Abstract title
\abstitle{Bat rescue and research: weaknesses and opportunities}

%% Author names
\absauthors{V. \textsc{Studer}$^1$, F. \textsc{Manzia}$^1$, F. \textsc{Renzopaoli}$^1$, A. \textsc{Tomassini}$^2$, L. \textsc{Ancillotto}$^3$}

\absaddress{$^1$Centro Recupero Fauna Selvatica Lipu di Roma, via Ulisse Aldrovandi 2, 00197 Roma\break
$^2$Tutela Pipistrelli – ONLUS, via Lodovico Bertonio, 20 – 00126 Roma\break
$^3$Wildlife Research Unit, Laboratorio di Ecologia Applicata, Sezione di Biologia e Protezione dei Sistemi Agrari e Forestali, Dipartimento di Agraria, Università degli Studi di Napoli Federico II, via Università 100, 80055 Portici (Napoli), Italy}

%% Abstract text
\abstext{
%% Author names for index. State each author separately using \index{Doe J.}
\index{Studer V.}
\index{Manzia  F.}
\index{Renzopaoli F.}
\index{Tomassini A.}
\index{Ancillotto L.}
%% The actual abstract text goes here
Wildlife rescue centres admit thousands of wild animals worldwide each year. The role of rescue centres in improving individual animal welfare is evident, as well as in educating people towards the importance of wildlife and how to deal with it; on the contrary, the opportunity to gather large amounts of valuable data, e.g. on wildlife presence, phenology and diseases, is often neglected, particularly in Italy. Among mammals, bats are the most frequent species admitted in rescue centres, thus they represent good candidates for wildlife research. Here we review the actual role of wildlife rescue centres in bat research and discuss their potentiality and weaknesses. We report the experience of LIPU’s wildlife rescue centre in Rome (Italy) as a case study, discussing its bat-related activities during the last 5 years. We highlight the importance of cooperation between rescue workers and researchers during all the rescue activities, from bat identification and admission recording to captive management, in order to obtain high-quality data.
} %% remember to close the abstract text block brace!!