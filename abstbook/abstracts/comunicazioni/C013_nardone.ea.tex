% Abstract file structure example : 
% \abstitle{title here}
% \absauthors{names and superscripts for affiliations here}
% \absaddress{affiliations, starting each one with its superscripts, separate affiliations with a \break}
% \abstext{
% \index{author abbreviated name, to be placed in authors' index}
% \index{create an index entry for each author}
%  The abstract text
% }

%% Abstract title
\abstitle{A first approach to the phylogeography of Daubenton's bat \emph{Myotis daubentonii}}

%% Author names
\absauthors{V. \textsc{Nardone}$^1$, D. \textsc{Russo}$^{1,2}$ C. \textsc{Ibañez}$^3$, J. \textsc{Juste}$^3$}

\absaddress{$^1$Wildlife Research Unit, Laboratorio di Ecologia Applicata, Dipartimento di Agraria. Università degli Studi di Napoli Federico II. Via Università 100, 80055 Portici (NA), Italy\break
$^2$School of Biological Sciences, University of Bristol, Bristol, UK\break
$^3$Estación Biólogica de Doñana, Consejo Superior de Investigaciones Científicas. Avenida Americo Vespucio, 41092 Sevilla, España}

%% Abstract text
\abstext{
%% Author names for index. State each author separately using \index{Doe J.}
\index{Nardone V.}
\index{Russo D.}
\index{Ibañez C.}
\index{Juste J.}
%% The actual abstract text goes here
The Daubenton's bat, \emph{Myotis daubentonii} (Vespertilionidae; Kuhl 1817) is a medium-size vespertilionid characterized by a wide Eurasian distribution. We explored the phylogeography of this bat looking at several European populations. The specific objectives were: to obtain the first accurate description of the hypervariable domains \textsf{I} and \textsf{II} for this bat; to identify the main lineages within the European populations; to analyze variation within and between the different lineages; to identify \emph{M. daubentonii} European glacial refugia.

We analyzed genetic diversity of the most informative markers at this evolutionary level such as the cytocrome b gene and the hypervariable domains \textsf{I} and \textsf{II} from the control region (D-loop) of the mtDNA of respectively 157 and 123 samples of \emph{M. daubentonii} from 63 locations of Europe. We found a great variability in repeated fragments of \textsf{HVI} and \textsf{HVII} and similarities corresponding to a common geographical origin. The data showed quite a remarkable differentiation with more than 50 different haplotypes. All phylogenetic reconstructions highlight the distinction of three very differentiated and highly structured main lineages: a lineage widespread in the Iberian Peninsula, in previous work identified on a morphological basis as \emph{M. d. nathalinae} (hereafter termed Iberian lineage); a lineage found in Italy, France, Switzerland, Germany, Sweden and in the Central and Northern Iberia (hereafter termed Italian lineage); and another lineage consisting of samples from Serbia, Montenegro, Greece, Netherlands, the north of Spain and Portugal (hereafter termed Balkan lineage). In the last two lineages, the Cyt \textit{b} fragment showed higher genetic variability within the Italian and Balkan peninsulas than within the rest of their geographic areas outside the peninsulas. We found the highest value of K2P genetic distance (0.93\%) and shared intermediate haplotypes by subgroups were  absent within the Iberian clade. In fact the parsimony network showed no haplotype shared by subgroups, suggesting a possible pattern of refugia-within-refugia in Iberia. The estimated demographic indices (\textsf{Fs}, \textsf{R\textsubscript{2}}, \textsf{D}) were all non-significant with values other than 0 and mismatching distribution multimodal for the Iberian lineage, indicating that there has been no expansion for this lineage. On the contrary, possible past events of demographic expansions were supported by the genetic variation pattern for the Balkan lineage, showing a negative and statistically significant \textsf{Fs} index and a significant \textsf{r} value of mismatching distribution. This distribution appeared smooth and unimodal, as expected in cases of rapid population expansion, for the Italian lineage. For this lineage, although the \textsf{r} statistic was not statistically significant, neutrality tests supported demographic expansion events in the recent past.

Our results demonstrated that Mediterranean Peninsulas (Italy, Iberia, Balkan) acted as glacial refugia for \emph{M. daubentonii}. The species' European populations have originated from the postglacial Palaearctic expansions of Italian and Balkan lineages, while the Iberian lineage did not cross the Pyrenees with a possible pattern of refugia-within-refugia as a consequence of the climatic cycles from the Pleistocene.
} %% remember to close the abstract text block brace!!