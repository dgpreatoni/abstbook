% Abstract file structure example : 
% \abstitle{title here}
% \absauthors{names and superscripts for affiliations here}
% \absaddress{affiliations, starting each one with its superscripts, separate affiliations with a \break}
% \abstext{
% \index{author abbreviated name, to be placed in authors' index}
% \index{create an index entry for each author}
%  The abstract text
% }

%% Abstract title
\abstitle{Studio sui chirotteri troglofili della Grotta di Calafarina (Pachino, SR, Sicilia sud-orientale)}

%% Author names
\absauthors{M. \textsc{Mucedda}$^1$,  G. \textsc{Fichera}$^2$, E. \textsc{Pidinchedda}$^1$}

\absaddress{$^1$Centro Pipistrelli Sardegna, Via G. Leopardi 1 – 07100 Sassari, Italy - batsar@tiscali.it\break
$^2$Universität Trier Universitätsring,15 - D-54286  Trier, Germany}

%% Abstract text
\abstext{
%% Author names for index. State each author separately using \index{Doe J.}
\index{Mucedda M.}
\index{Fichera G.}
\index{Pidinchedda E.}
%% The actual abstract text goes here
La Grotta di Calafarina è una cavità naturale situata nel comune di Pachino (SR), che ospita nel suo interno una grande colonia di pipistrelli troglofili, che la rendono una delle più importanti della Sicilia.

Nel corso del biennio 2004-2005 è stato effettuato uno studio, con monitoraggio periodico della popolazione di chirotteri all’interno della grotta, che è stato ripetuto nel triennio 2011-2013, allo scopo di stabilire quali specie siano presenti, valutare l’entità numerica e ricostruire il loro ciclo biologico annuale.

Il nostro studio ha consentito di accertare nella grotta la presenza di 5 specie di chirotteri: \emph{Rhinolophus ferrumequinum}, \emph{Rhinolophus mehelyi}, \emph{Myotis myotis},\emph{ Myotis capaccinii} e \emph{Miniopterus schreibersii}. La maggior parte dei pipistrelli non utilizza la grotta tutto l’anno, ma essi compiono dei movimenti migratori che li portano nella cavità generalmente in primavera, per poi gradualmente abbandonarla in autunno.

Durante il periodo invernale permangono nella grotta solo poche decine di pipistrelli, poiché la cavità ha elevate temperature interne e risulta poco idonea per il letargo. 

In primavera la popolazione cresce notevolmente, con l’arrivo di molte centinaia di pipistrelli in migrazione, che si radunano in un nicchione del soffitto dell’ampia sala terminale, dove formano una grande colonia plurispecifica di riproduzione, in cui le nascite hanno inizio nel mese di maggio. La colonia raggiunge numericamente circa 2000 esemplari e risulta formata in massima parte da \emph{Myotis myotis} e da \emph{Miniopterus schreibersii}, che sono le specie preponderanti, e da un numero molto ridotto di \emph{Myotis capaccinii} e di \emph{Rhinolophus mehelyi}. Il \emph{Rhinolophus ferrumequinum} è stato osservato invece in pochi esemplari e non si aggrega alle altre specie che formano la colonia di riproduzione

In autunno i pipistrelli pian piano abbandonano la grotta diretti alle località di svernamento, riducendosi in novembre a qualche centinaio di esemplari.

La colonia riproduttiva della Grotta di Calafarina è una delle più numerose della Sicilia e geograficamente risulta essere la più meridionale d'Italia. 
Non sono emerse variazioni di rilievo tra le osservazioni del 2004--2005 e quelle del 2011--2013, per cui si ritiene che a distanza di 7--8 anni la popolazione di chirotteri della Grotta di Cala Farina sia stabile e quindi non soffra di particolari pressioni o minacce dirette.  
Sono invece emerse notevoli differenze rispetto alle osservazioni fatte da Bruno Ragonese negli anni ’60 del secolo scorso. All’epoca infatti non risultavano presenti le due specie \emph{R. mehelyi} e \emph{M. schreibersii}, che evidentemente si sono stabilite nella grotta solo in tempi più recenti. Nell’arco di 50 anni quindi la popolazione dei chirotteri ha subito importanti modifiche nella sua composizione specifica. 
Particolarmente rilevante dal punto di vista protezionistico è la presenza e riproduzione del \emph{Rhinolophus mehelyi}, di cui in tutta la Sicilia sono note attualmente due sole stazioni, e che è talmente ridotto numericamente da risultare minacciato di estinzione nella regione.
} %% remember to close the abstract text block brace!!