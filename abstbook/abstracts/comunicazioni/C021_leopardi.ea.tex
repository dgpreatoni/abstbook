% Abstract file structure example : 
% \abstitle{title here}
% \absauthors{names and superscripts for affiliations here}
% \absaddress{affiliations, starting each one with its superscripts, separate affiliations with a \break}
% \abstext{
% \index{author abbreviated name, to be placed in authors' index}
% \index{create an index entry for each author}
%  The abstract text
% }

%% Abstract title
\abstitle{Bat lyssaviruses circulation in European bats: is there an actual conflict between species conservation and public health?}

%% Author names
\absauthors{S. \textsc{Leopardi}$^1$, P. \textsc{Priori}$^{2,3}$, D. \textsc{Scaravelli}$^{2,4}$, B. \textsc{Zecchin}$^1$, G. \textsc{Cattoli}$^1$, P. \textsc{De Benedictis}$^1$}

\absaddress{$^1$FAO and National Reference Centre for Rabies, OIE Collaborating Centre for Diseases at the Animal/Human Interface, Istituto Zooprofillatico Sperimentale delle Venezie, Viale dell’Università 10, 35020 Legnaro (PD), Italy\break
$^2$S.T.E.R.N.A. \& Museo Ornitologico ``F.Foschi'', via Pedrali 12, 47100, Forlì, Italy\break
$^3$Department of Earth, Life and Environmental Sciences (DiSTeVA), Campus Scientifico Enrico Mattei, via Cà Le Suore, 2/4,  61029 Urbino, Italy\break
$^4$Laboratory of Pathogens’ Ecology, Department of Veterinary Medical Sciences, University of Bologna, via Tolara di Sopra 50, 40064 Ozzano Emilia (Bologna), Italy}

%% Abstract text
\abstext{
%% Author names for index. State each author separately using \index{Doe J.}
\index{Leopardi S.}
\index{Priori P.}
\index{Scaravelli D.}
\index{Zecchin B.}
\index{Cattoli G.}
\index{De Benedictis P.}
%% The actual abstract text goes here
In the last decades, numerous bat species have been identified as possible reservoirs for emerging infectious diseases such as Ebola and SARS-like viruses. The interest in these animals has been growing ever since, to the point that a great number of novel viruses have been discovered in different species worldwide, whose zoonotic potential is still to be fully assessed. A huge variety of viruses has also been found in European bats. Among these, rabies-related lyssaviruses (RRLVs) are the only proven zoonotic pathogens. Rabies is a viral encephalomyelitis causing about 59.000 deaths every year worldwide. The virus is transmitted by animal bites and affects the central nervous system (CNS): once the symptoms develop, rabies is nearly always fatal. While rabies virus (RABV) in Europe is still mostly associated with non-flying mammals, at least five different lyssaviruses are currently known to infect European bats. Among them, the European bat Lyssaviruses 1 and 2 (EBLV1 and EBLV2) are certainly the most widely distributed. In particular, EBLV1 has been associated to the bat population in mainland Europe and EBLV2 in the United Kingdom and the Netherlands, with only few cases reported in Finland, Germany and Switzerland. Both EBLV1 and 2 showed to be less pathogenic than RABV in animal experimental models, with a possible benign outcome reported in bats. Although the vast majority of EBLV1 have been isolated from the serotine bat \emph{(Eptesicus serotinus)} and EBLV2 is most commonly associated with the Daubenton’s bat \emph{(Myotis daubentonii)}, the role of different species in the epidemiology of both viruses cannot be excluded. Virological evidence of other bat LYSVs is sporadic and restricted to defined geographic areas, as for the West Caucasian bat lyssavirus (WCBL) identified in Russia, the Bokeloh Bat Virus (BBV) in Germany and France and the putative Lleida Bat Virus (LLEBV), for which genetic evidence but not virus isolation has been proven in a Spanish bat.

Although spillover events to humans have been documented for EBLV-1 and EBLV-2 only, LYSVs are all potentially able to cause a fatal rabies-related encephalomyelitis in mammals, humans included. Furthermore, sporadic EBLV spillover cases to both domestic and wild animals have been reported in Europe, thus posing a potential risk for possible host-jump to non-flying mammals. For all these reasons, the international authorities consider bat associated LYSVs as a public health issue and have recommended the implementation of harmonized surveillance schemes at a European level. While both active and passive surveillance for bat rabies are in place in western European countries, such an implementation is still very heterogeneous in the entire continent and based on limited and opportunistic sampling, whose results have no epidemiological significance. Main challenges are related to the apparent conflict between bat conservation and public health. However, no negative outcome for bat populations has been reported from European countries where bat lyssaviruses have been identified, nor related to sampling or control measures. This includes countries such as the United Kingdom, where human cases have occurred. Implemented control plans included access restriction measures in some maternity colonies in Spain, which determined a limited disturbance other than spillover risks.

In Italy, although an active monitoring approach has provided evidence for bat LYSV circulation, passive surveillance strategies have not been enhanced yet. Samples analyzed in the past years are limited, often collected from non-target species, and the circulation of bat LYSV has not been demonstrated by virological evidence. Following the lead from other European and non-European countries, the interest for the role of these animals in public health is currently increasing also in Italy, with new projects funded for future investigations, as they will contribute to increase our knowledge about bat LYSV in Italy.

\vskip3mm
\begin{footnotesize}
\textbf{Acknowledgments}: authors wish to thank the Italian Ministry of Health for partially funding this study (GR2011-02350591, AN EPIZOOTIOLOGICAL SURVEY OF BATS AS RESERVOIRS OF EMERGING ZOONOTIC VIRUSES IN ITALY: IMPLICATIONS FOR PUBLIC HEALTH AND BIOLOGICAL CONSERVATION).
\end{footnotesize}
} %% remember to close the abstract text block brace!!