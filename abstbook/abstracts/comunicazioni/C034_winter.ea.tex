% Abstract file structure example : 
% \abstitle{title here}
% \absauthors{names and superscripts for affiliations here}
% \absaddress{affiliations, starting each one with its superscripts, separate affiliations with a \break}
% \abstext{
% \index{author abbreviated name, to be placed in authors' index}
% \index{create an index entry for each author}
%  The abstract text
% }

%% Abstract title
\abstitle{Occurrence and activity patterns of bats in different habitat types in the northern part of the island of Asinara (Sardinia, Italy)}

%% Author names
\absauthors{R. \textsc{Winter}$^1$, J. \textsc{Treitler}$^1$, U. \textsc{Kierdorf}$^1$, S. \textsc{Schmidt}$^2$, J. \textsc{Mantilla-Contreras}$^1$}

\absaddress{$^1$Institute of Biology and Chemistry, University of Hildesheim, Germany, email: \url{winterr@uni-hildesheim.de}\break
$^2$Institute of Zoology, University of Veterinary Medicine Hannover Foundation, Germany}

%% Abstract text
\abstext{
%% Author names for index. State each author separately using \index{Doe J.}
\index{Winter R.}
\index{Treitler J.}
\index{Kierdorf U.}
\index{Schmidt S.}
\index{Mantilla-Contreras J.}
%% The actual abstract text goes here
The Mediterranean region is a biodiversity hotspot for bats in Europe, which are subject to a high risk of habitat loss due to the diverse anthropogenic impacts. To initiate conservation measures it is essential to identify the most important sites for hunting and roosting. Protected areas like National Parks play an important role for the conservation of suitable habitat structures.
 
The Italian island and National Park Asinara is characterized by low vegetation shaped by a large number of grazing animals and includes many abandoned buildings from former settlements. Apart from a small area, no forested areas are present. To analyze if and how bats use those particular structures, we studied the abundance and activity patterns of bats in different habitat types (``forest'', ``semi-open'', ``open'' and ``settlement''). For each type of habitat, four representative study sites were selected on the northern part of the island and recordings were done from June until the end of August 2013. Each night the bat activity was recorded on a different study site with a bat detector (EM3+). Additionally, we captured insects to study if bat activity correlated with the amount of prey.

We identified nine species in the study area, including many recordings of the endangered \emph{Rhinolophus hipposideros}. Bat activity in terms of bat passes, and social calls, was very high in the study area. Activity was higher in ``forest'' and ``settlement'' compared to the more open habitats. There was no effect of prey amount or abiotic parameters on activity. In contrast, feeding activity, determined by feeding buzzes, was lightly influenced by prey amount. Thus bats preferred certain habitat types and adapted their hunting behavior to prey availability. Although the most affected habitat type ``forest'' covers only a small area of the island, the high bat activity and number of species in the whole study area on Asinara emphasizes its role for implementing the habitat directives for bats in Europe.
} %% remember to close the abstract text block brace!!