% Abstract file structure example : 
% \abstitle{title here}
% \absauthors{names and superscripts for affiliations here}
% \absaddress{affiliations, starting each one with its superscripts, separate affiliations with a \break}
% \abstext{
% \index{author abbreviated name, to be placed in authors' index}
% \index{create an index entry for each author}
%  The abstract text
% }

%% Abstract title
\abstitle{Bat bioacoustic studies in Buenos Aires province, Argentina}

%% Author names
\absauthors{G. \textsc{Giacomini}$^1$, P. \textsc{Priori}$^2$, D. \textsc{Scaravelli}$^3$}

\absaddress{$^1$Science and Management of Nature Master, School of Science, University of Bologna, Italy, email: \url{giada.giacomini@studio.unibo.it}\break
$^2$Department of Earth, Life and Environmental Sciences, University of Urbino, Scientific Campus E. Mattei, via Cà Le Suore 2/4, 61029 Urbino, Italy, email: \url{pamela.priori@uniurb.it}\break
$^3$Department of Veterinary Medical Sciences, via Tolara di sopra 50, Ozzano Emilia (BO), email: \url{dino.scaravelli@unibo.it}}

%% Abstract text
\abstext{
%% Author names for index. State each author separately using \index{Doe J.}
\index{Giacomini G.}
\index{Priori P.}
\index{Scaravelli D.}
%% The actual abstract text goes here
The knowledge of echolocation in Argentinian bats is poor. Here we present a first attempt to provide a description of echolocation calls of three species in the family Molossidae (\emph{Eumops bonariensis}, \emph{Molossus molossus}, \emph{Tadarida brasiliensis}) and three from the Vespertilionidae (\emph{Myotis albescens}, \emph{M. dinellii}, \emph{M. levis}). For each species we measured 12 echolocation parameters with the Batsound software: frequency peak (Fpeak), minimum frequency (Fmin), maximum frequency (Fmax), bandwidth (BW), harmonics frequency peak (HFpeak), start frequency (Fstart), end frequency (Fend), low frequency (Flow), call duration (cd), slope (S) and duty cycle (DC). The 3 molossid species are separable mostly by Fpeak: \emph{E. bonariensis} (n=6 calls) presents the lowest Fpeak 19.5$\pm$0.4 kHz, \emph{M. molossus} shows a bimodal call pattern with the first call frequency peaking at 31.6$\pm$2.3 kHz (n=21 calls) and the second call 41.1$\pm$1.3 kHz (n=24 calls); finally \emph{T. brasiliensis} has unimodal calls of 31.4$\pm$1.9 kHz Fpeak. We also observed a particular ``hook'' (FM-up) at the beginning of calls of \emph{T. brasiliensis} emerging from the roost. This feature was already pointed out in a Brazilian free-tailed bat population in Texas emerging from their roost. The same pattern is also found in \emph{M. molossus} diring emergence suggesting a possible role for social communication in the two species. The echolocation calls of emerging \emph{M. molossus} are similar to those emitted by hand-released bats for all 12 parameters, i.e. there is an increasein the FM component of the call, typical call structure in  cluttered habitat and quite different from the search phase call shape. The \emph{Myotis} species characterisation is more difficult because of their cryptic morphology and converging echolocation. We performed an ANOVA on call parameters extracted with the semi-automatic software Raven. This analysis shows how the two sibling species, \emph{M. levis} and \emph{M. dinellii}, are almost completely separated based on HFpeak and Delta frequency (difference between the upper and lower frequency limits of the selection). \emph{M. albescens} and \emph{M. dinellii} have similar echolocation parameters but they are morphologically different. Although a genetic analysis is still needed for the Argentinian \emph{Myotis} species to clarify their taxonomic status, this result supports the hypothesis that \emph{M. levis} and \emph{M. dinellii} are two species and not subspecies as strongly believed until the 2006. 

We also studied the ecology of the bat community at different locations. Because no information on echolocation calls is available for the study sites, calls were categorised as sonotypes, rather than classified as species, for a total of 7 units. These sonotypes were chosen in relation to the sonogram (shape of the calls) and 8 parameters of the calls. In la Plata city, from early fall to the beginning of the winter, a peak activity was found at the beginning of May (fall) and a decreasing trend until mid-June (beginning of winter). The bat community shows changes along the seasons: in early fall 6 sonotypes were present, then three gradually disappear. Because the variation of the species' echolocation repertoire is unknown, it is unclear whether such changes should be interpreted as a decrease in number of species and/or in the number of call types broadcast, maybe in relation with social behaviour. Is interesting to note that in the beginning of winter only 3 sonotypes were recorded, all belonging to molossid species (Sonotype 1: Fpeak 28.5 kHz, BW 4.4 ms, cd 13.9 ms; Sonotype 2: Fpeak 18.1 kHz, BW 4.0 ms, cd 19.3 ms; Sonotype 3: Fpeak 24.4 kHz, BW 3.2 kHz, cd 15.4 ms) where Sonotype 2 likely belongs to \emph{Eumops} sp.. Furthermore in the beginning of winter the activity over the night seems to be constantly of a high intensity since 2 to 4 hours after the sunset, while in fall it shows two peaks: one 1--2 hours after the sunset and another 3--5 hours after the sunset.  
} %% remember to close the abstract text block brace!!