% Abstract file structure example : 
% \abstitle{title here}
% \absauthors{names and superscripts for affiliations here}
% \absaddress{affiliations, starting each one with its superscripts, separate affiliations with a \break}
% \abstext{
% \index{author abbreviated name, to be placed in authors' index}
% \index{create an index entry for each author}
%  The abstract text
% }

%% Abstract title
\abstitle{Strategie di conservazione dei chirotteri negli affioramenti gessosi dell’Emilia-Romagna: progetto LIFE+ \textit{Gypsum}}

%% Author names
\absauthors{A. \textsc{Ruggieri}$^1$, T. \textsc{Mondini}$^2$, A. \textsc{Peron}$^2$, F. \textsc{Suppini}$^2$, F. \textsc{Grazioli}$^3$}

\absaddress{$^1$Museo Civico di Storia Naturale di Piacenza, via Scalabrini 107, 29121 Piacenza\break
$^2$Naturale s.n.c. di Fabio Suppini \& C., via J.F. Kennedy 1, 40024 Castel San Pietro Terme\break
$^3$Gruppo Speleologico Bolognese - Unione Speleologica Bolognese, Piazza VII Novembre 1944 n. 7, 40122 Bologna}

%% Abstract text
\abstext{
%% Author names for index. State each author separately using \index{Doe J.}
\index{Ruggieri A.}
\index{Mondini T.}
\index{Peron A.}
\index{Suppini F.}
\index{Grazioli F.}
%% The actual abstract text goes here
In Emilia Romagna le aree carsiche gessose costituiscono circa l’1\% del territorio e comprendono un mosaico di habitat d’interesse comunitario strettamente interconnessi. Nelle evaporiti triassiche e messiniane della regione sono presenti differenti e caratteristici aspetti del carsismo superficiale e sotterraneo con i maggiori sistemi carsici attualmente noti in Italia. Alle aree carsiche e, in particolare, all’habitat delle grotte, sono associate molte specie di chirotteri.

Le specie di chirotteri che utilizzano le cavità sotterranee come rifugio nei periodi critici dell’ibernazione e della riproduzione sono particolarmente sensibili al disturbo antropico. Tale disturbo indesiderato può essere evitato grazie all’apposizione di inferriate agli ingressi.

Nel 2010 con il Progetto LIFE+/08/NAT/IT/000369 ``\textit{Gypsum}'' è stato avviato uno studio sulla chirotterofauna in 6 Siti di Interesse Comunitario (Direttiva Habitat) dell’Emilia-Romagna, caratterizzati da carsismo su formazioni gessose: IT4030017 Ca’ del Vento, Ca’ del Lupo, gessi di Borzano (Reggio Emilia), IT4030009 Gessi triassici (Reggio Emilia), IT4050001 Gessi bolognesi, calanchi dell’Abbadessa (Bologna), IT4050027 Gessi di Monte Rocca, Monte Capra e Tizzano (Bologna), IT4070011 Vena del gesso romagnola (Bologna-Ravenna) e IT4090001 Onferno (Rimini).

L’indagine \textit{ante operam} (realizzata nel periodo: estate 2010 -- inverno 2011) ha contribuito ad aumentare la conoscenza della chirotterofauna troglofila sotto l’aspetto qualitativo e quantitativo, a comprendere nel dettaglio la fenologia circa l’uso dei rifugi ipogei e a progettare gli interventi di protezione di grotte naturali e cavità artificiali.

Per il monitoraggio sono state adottate diverse tecniche: ispezione delle cavità sotterranee con conta a vista o su fotografia (per colonie numerose), rilevamento bioacustico con bat detector in prossimità degli ingressi e nei principali ambienti di foraggiamento/abbeverata (nel raggio di circa 5 km dai siti di rifugio), acquisizione d’immagini con foto trappola IR e riprese video all’infrarosso che riprendono gli esemplari in uscita, rilevamento dei flussi di transito tramite barriere fotoelettriche con conta elettronica.

Per il monitoraggio invernale complessivamente sono state controllate 46 cavità: 18 nei Gessi Bolognesi, 3 nei Gessi di Zola Predosa, 1 nei Gessi di Onferno, 12 nella Vena del Gesso Romagnola, 6 nei Gessi Triassici e 6 nei Gessi Messiniani.

Altre cavità sono state controllate nel periodo estivo, alla ricerca di rifugi e colonie di riproduzione.

Nei siti interessati dal progetto sono presenti almeno 19 specie, tutte incluse nella Direttiva Habitat (8 sono incluse nell’allegato II).
Per le specie \textit{target} è stato effettuato il conteggio degli esemplari svernanti; complessivamente risultano: \emph{Rhinolophus ferrumequinum} 1741 esemplari, \emph{R. hipposideros} 472 esemplari, \emph{R. euryale} 1165 esemplari, gruppo \emph{Myotis myotis}/\emph{M. blythii} 23, \emph{Miniopterus schreibersii} 12445 esemplari.

La selezione delle cavità ipogee, da sottoporre a tutela con il posizionamento di un cancello, è stata stabilita in base alla presenza e all’abbondanza numerica delle specie \textit{target}.

Ad oggi sono state  chiuse 19 cavità (18 grotte naturali e 1 cava  di gesso abbandonata) e sono  in corso i lavori di chiusura di altre 2 cavità naturali. Per alcune cave di gesso abbandonate sono in itinere accordi con i proprietari.

Con l’indagine \textit{post operam} (autunno 2014 -- primavera 2015) è stata valutata la permeabilità delle inferriate ai chirotteri e il riutilizzo delle cavità già sottoposte a chiusura.

Il monitoraggio e i lavori d’installazione dei cancelli sono stati eseguiti operando in stretta collaborazione con gli speleologi dei gruppi locali che hanno contribuito in modo volontario.

Altre azioni concernenti la conservazione sono state il posizionamento di bat box e l’acquisto di terreni privati.
} %% remember to close the abstract text block brace!!