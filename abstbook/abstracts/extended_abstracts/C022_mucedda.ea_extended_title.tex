% Abstract file structure example : 
% \abstitle{title here}
% \absauthors{names and superscripts for affiliations here}
% \absaddress{affiliations, starting each one with its superscripts, separate affiliations with a \break}
% \abstext{
% \index{author abbreviated name, to be placed in authors' index}
% \index{create an index entry for each author}
%  The abstract text
% }

%% Abstract title
\abstitle{Note sui pipistrelli nelle piccole isole della Sardegna}

%% Author names
\absauthors{M. \textsc{Mucedda}, E. \textsc{Pidinchedda}, M.L. \textsc{Bertelli}}

\absaddress{Centro Pipistrelli Sardegna, Via G. Leopardi 1 – 07100 Sassari, Italy - \texttt{batsar@tiscali.it}}

%% Abstract text
\abstext{
%%% Author names for index. State each author separately using \index{Doe J.}
\index{Mucedda M.}
\index{Pidinchedda E.}
\index{Bertelli M.L.}
%%% The actual abstract text goes here
\small
\textbf{Riassunto}\\
È stato realizzato uno studio sui chirotteri nelle piccole isole della Sardegna, tendente a stabilire quali specie siano presenti. Oggetto della ricerca sono state 15 isole, a partire da nord: La Maddalena, Caprera, Santo Stefano, Spargi, Budelli, Santa Maria, Tavolara, Molara, Figarolo, Asinara, Piana, San Pietro, Sant’Antioco, Serpentara e Cavoli. Su 21 specie di chirotteri presenti nella Sardegna, almeno 11 sono state riscontrate nel totale delle isole minori: \emph{Rhinolophus ferrumequinum}, \emph{Rhinolophus hipposideros}, \emph{Myotis capaccinii}, \emph{Myotis daubentonii}, \emph{Miniopterus schreibersii}, \emph{Pipistrellus pipistrellus}, \emph{Pipistrellus kuhlii}, \emph{Hypsugo savii}, \emph{Pipistrellus pygmaeus}, \emph{Tadarida teniotis}, \emph{Eptesicus serotinus}/\emph{Nyctalus leisleri}.
 
Il più alto numero di specie di pipistrelli si riscontra all’Asinara con 10 entità, seguita da Caprera e Tavolara con 8, quindi La Maddalena con 7. Nelle altre isole, soprattutto in quelle più piccole, il numero diminuisce sino al minimo di una sola specie. Sulla base dei dati presenti in bibliografia, L’Asinara si attesta in cima alle isole italiane con il maggior numero di specie alla pari con l’Isola d’Elba.  Le specie più ampiamente diffuse sono \emph{Pipistrellus pipistrellus}, presente in tutte le isole, seguita da \emph{Tadarida teniotis} in 12 isole, \emph{Hypsugo savii} e \emph{Pipistrellus kuhlii} in 9 isole. La più rara è risultata invece \emph{Myotis daubentonii} osservata in una sola isola.\\

{\footnotesize Parole chiave: pipistrelli; Sardegna; piccole isole}

\columnbreak

\textbf{Abstract}\\
The present study was carried out on bats on the small islands of Sardinia, with an aim of determining which species were present. The study focused on 15 islands, beginning from north: La Maddalena, Caprera, Santo Stefano, Spargi, Budelli, Santa Maria, Tavolara, Molara, Figarolo, Asinara, Piana, San Pietro, Sant'Antioco, Serpentara and Cavoli.

Of the 21 species of bats found in Sardinia, at least 11 were found on the small islands: \emph{Rhinolophus ferrumequinum}, \emph{Rhinolophus hipposideros}, \emph{Myotis capaccinii}, \emph{Myotis daubentonii}, \emph{Miniopterus schreibersii}, \emph{Pipistrellus pipistrellus}, \emph{Pipistrellus kuhlii}, \emph{Hypsugo savii}, \emph{Pipistrellus pygmaeus}, \emph{Tadarida teniotis}, and \emph{Eptesicus serotinus}/\emph{Nyctalus leisleri}.

The highest number of species of bats was found on Asinara with 10 entities, followed by Caprera and Tavolara with eight each, then La Maddalena with seven. On the other islands, especially the smaller ones, the number decreased to a minimum of one species. Based on the data in the literature, Asinara is attested with the Elba Island on top of the Italian islands with the greatest number of species.

The species most widely spread were \emph{Pipistrellus pipistrellus}, which were present on all islands, followed by \emph{Tadarida teniotis} on 12 islands, and \emph{Hypsugo savii} and \emph{Pipistrellus kuhlii} on nine islands. The rarest was \emph{Myotis daubentonii} observed on only one island.\\

{\footnotesize Keywords: bats; sardinia; small islands}

} %% remember to close the abstract text block brace!!