% Abstract file structure example : 
% \abstitle{title here}
% \absauthors{names and superscripts for affiliations here}
% \absaddress{affiliations, starting each one with its superscripts, separate affiliations with a \break}
% \abstext{
% \index{author abbreviated name, to be placed in authors' index}
% \index{create an index entry for each author}
%  The abstract text
% }

%% Abstract title
\abstitle{Studio sui chirotteri troglofili della Grotta di Calafarina (Pachino, SR, Sicilia sud-orientale)}

%% Author names
\absauthors{M. \textsc{Mucedda}$^1$,  G. \textsc{Fichera}$^2$, E. \textsc{Pidinchedda}$^1$}

\absaddress{$^1$Centro Pipistrelli Sardegna, Via G. Leopardi 1 – 07100 Sassari, Italy - \texttt{batsar@tiscali.it}\break
$^2$Department of Biogeography, Trier University, Universitätsring,15 - D-54286  Trier, Germany}

%% Abstract text
\abstext{
%%% Author names for index. State each author separately using \index{Doe J.}
\index{Mucedda M.}
\index{Fichera G.}
\index{Pidinchedda E.}
%%% The actual abstract text goes here
\label{ext:E018}
\small
\textbf{Riassunto}\\
In questo lavoro gli autori espongono i risultati di un studio effettuato nel 2004--2005 e nel 2011--2013 nella Grotta di Calafarina (Pachino, SR), che è una delle cavità più importanti della Sicilia per la sua popolazione di chirotteri. Nella grotta è presente una colonia polispecifica di molte centinaia di pipistrelli, costituita in massima parte da \emph{Myotis myotis} e \emph{Miniopterus schreibersii}, e in numero ridotto da \emph{Myotis capaccinii} e \emph{Rhinolophus mehelyi}. Si tratta di una colonia riproduttiva nella quale i pipistrelli si aggregano in primavera, partoriscono tra maggio e giugno e permangono nella grotta sino all’autunno, quando quasi tutti abbandonano la cavità. In periodo invernale invece permangono solo poche decine di pipistrelli. Osservati anche pochi esemplari di \emph{Rhinolophus ferrumequinum} che non si aggregano alla colonia \textit{nursery}. È questo sinora l’unico sito di riproduzione certo di \emph{Rhinolophus mehelyi} in Sicilia.\\

{\footnotesize Parole chiave: pipistrelli; grotta; Sicilia; \emph{Rhinolophus mehelyi}}

\columnbreak

\textbf{Abstract}\\
In this paper, the authors present the results of a study carried out from 2004 to 2005 and from 2011 to 2013 in the Cave of Calafarina (Pachino, SR), which is one of most important cavities in Sicily in terms of its bat population. In the cave, there is a polyspecific colony of many hundreds of bats, consisting mostly of \emph{Myotis myotis} and \emph{Miniopterus schreibersii}, with a few \emph{Myotis capaccini} and \emph{Rhinolophus mehelyi}. It is a breeding colony where bats congregate in the spring, give birth in May and June and remain in the cave until the autumn, when almost all bats leave the cave. In the winter, only a few dozen bats remain. A few specimens of \emph{Rhinolophus ferrumequinum}, which do not live in the nursery colony, were also observed. This is, until now, the only definite breeding site of \emph{Rhinolophus mehelyi} in Sicily.\\

{\footnotesize KeyWords: bats; cave; Sicily; \emph{Rhinolophus mehelyi}} 

} %% remember to close the abstract text block brace!!