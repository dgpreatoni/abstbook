% Abstract file structure example : 
% \abstitle{title here}
% \absauthors{names and superscripts for affiliations here}
% \absaddress{affiliations, starting each one with its superscripts, separate affiliations with a \break}
% \abstext{
% \index{author abbreviated name, to be placed in authors' index}
% \index{create an index entry for each author}
%  The abstract text
% }

%% Abstract title
\abstitle{La chirotterofauna dei boschi vetusti nel Parco Nazionale del Pollino}

%% Author names
\absauthors{P.P. \textsc{De Pasquale}}

\absaddress{Wildlife Consulting, Viale S. Mercadante 26, 70132 Bari, Italy.
Gruppo Italiano Ricerca Chirotteri, c/o Dipartimento di Biologia Animale, via Taramelli 24, I-27100 Pavia}

%% Abstract text
\abstext{
%% Author names for index. State each author separately using \index{Doe J.}
\index{De Pasquale P.P.}

%%% The actual abstract text goes here
\small
\textbf{Riassunto}\\
I boschi vetusti sono habitat particolarmente rari e ricchi di biodiversità e solo marginalmente sono interessati da eventi di disturbo, che ha consentito la loro colonizzazione da parte di taxa specializzati. Sono importanti per la conservazione di molte specie di chirotteri, che frequentano questi ambienti soprattutto come aree di foraggiamento e per l’elevata qualità e disponibilità dei siti roost. In questo lavoro vengono presentati i risultati di uno studio preliminare condotto nel territorio del Parco Nazionale del Pollino, utilizzando delle reti mist net per le catture temporanee e un bat detector in espansione temporale, per i rilievi ultrasonori. Lo studio ha permesso di compilare una checklist dei chirotteri forestali presenti nel Parco, di valutare l’attività dei pipistrelli in relazione a 4 tipologie di popolamenti vetusti, alle classi di vetustà, e di valutare le differenze nella distribuzione del sesso e delle classi di età, per tipologia forestale. I popolamenti vetusti selezionati sono: faggeta, bosco misto di faggio (\emph{Fagus sylvatica}) e abete bianco (\emph{Abies alba}), bosco misto di faggio e cerro (\emph{Quercus cerris}) e bosco misto con dominanza di aceri (\emph{Acer pseudoplatanus, A. opalus subsp. obtusatum, A. lobelii, A. campestre, A. Platanoides}). Le foreste oggetto di studio sono caratterizzate da una diversità vegetazionale e una eterogeneità strutturale che ha determinato un’elevata ricchezza in specie di chirotteri e gran parte di esse risulta minacciata o in pericolo di estinzione in Italia. Il mantenimento di una matrice costituita da varie tipologie di boschi vetusti nel comprensorio del Parco è fondamentale per la tutela dei chirotteri, per cui è necessario definire delle linee guida per una corretta gestione forestale.\\

{\footnotesize Parole chiave: boschi vetusti, diversità, \textit{bat detector}, ricchezza specifica, linee guida}

\columnbreak

\textbf{Abstract}\\
 Old-growth forests are habitats very rare, rich in biodiversity and only marginally affected by disturbance events, with highest levels of diversity and more specialized species. They are important for the conservation of many bats that use these habitats for feeding and especially for the high quality and availability of roost sites. This paper presents the results of a preliminary study conducted in the territory of Pollino National Park (southern Italy), using two methods (mist-netting and bat detector). The objectives of the study are: (i) to inventory the bat species in forest habitats, (ii) to compare bat activity and the sex and age class distributions of bats, between two types of old-growth forests. Stands selected are: beech (\emph{Fagus sylvatica}) and silver fir (\emph{Abies alba}) mixed stand, beech and oak (\emph{Quercus cerris}) mixed stand. Forests are characterized by a high structural heterogeneity with high species richness of bats and  many of them are threatened or endangered in Italy. Maintaining a matrix of different types of forest stands in the Park, is necessary to preserve bat populations and it is important define the guidelines for  bat conservation in forests.\\

Keywords: .
{\footnotesize Keywords: old-growth forests, diversity, bat detector, species richness, guidelines.} 
} %% remember to close the abstract text block brace!!