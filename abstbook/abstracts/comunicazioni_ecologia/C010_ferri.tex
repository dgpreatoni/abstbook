% Abstract file structure example : 
% \abstitle{title here}
% \absauthors{names and superscripts for affiliations here}
% \absaddress{affiliations, starting each one with its superscripts, separate affiliations with a \break}
% \abstext{
% \index{author abbreviated name, to be placed in authors' index}
% \index{create an index entry for each author}
%  The abstract text
% }

%% Abstract title
\abstitle{Bats of Reatini Mountains: diversity and abundance by elevation (Central Italy, NE Latium)}

%% Author names
\absauthors{V. \textsc{Ferri}}

\absaddress{L.E.S.A., Department of Biology, University of Rome 2 ``Tor Vergata''. E-mail: \url{vincenf@tin.it}}

%% Abstract text
\abstext{
%% Author names for index. State each author separately using \index{Doe J.}
\index{Ferri V.}
%% The actual abstract text goes here
 During three years, 2012--2014, we surveyed 23 bat detection sites located in the territory of SPA IT6020005 Monti Reatini, ranging from 954 to 1890 m a. s. l., within  four main  habitat types (major water bodies, oak forests, beech forests, permanent pastures). Every site was preselected at five different elevation bands (900--1100; 1100--1300; 1300--1500; 1500--1700; 1700--1900). Each elevation band was sampled once a month  during  May, June, July and August of 2012--2014, with a total of four nights of sampling for each site. Three acoustic detectors (1 D240x and 2 D1000x Pettersson), were used as automatic recorder units and placed in on a 3 meter pole, from sunset until 05:00. Bat search calls, \textsf{BC}, were analyzed using BatSound 4.03. 19 species groups of bats were positively identified at the 23 sampling sites, with 4 woodland bat species  (\emph{Barbastella barbastellus}, \emph{Myotis bechsteinii}, \emph{M. mystacinus} and \emph{Pipistrellus pygmaeus}). \textsf{BC} were used to determine species diversity, \textsf{S},  and activity levels, \textsf{frBC}, within each elevation band. The highest \textsf{BC} was contained in the 1500--1700 band and \textsf{S} was significantly highest in water bodies, \textsf{WB} ($\chi^2$=41.509, p<0.0001). \textsf{WB} were used from all species but were especially important to \emph{Myotis}  bats, \emph{Hypsugo savii} and four species of  \emph{Pipistrellus} (\emph{kuhlii}, \emph{nathusii}, \emph{pygmaeus} and \emph{pipistrellus}). Beech and oak forests were used in different ways: in general not selected significantly the first and actively selected the second. Regarding \textsf{frBC} by elevation between most abundant species (> 100 \textsf{BC})  \emph{Pipistrellus pipistrellus} was the most dissimilar in terms of relative frequencies than the others. Bats community of the Study Area consists of a large and ecologically diverse set of species characteristic of agroforestry mosaics systems included in Central Apennine mountains with 3 dominant groups: \emph{Hypsugo savii},  \emph{Pipistrellus kuhlii} \& \emph{P. nathusii}, \emph{Pipistrellus pipistrellus}, while \emph{Rhinolophus  ferrumequinum} and \emph{R. hipposideros} were sampled with rather low densities, but as in other situation, presence of these \emph{Rhinolophidae} and of \emph{Plecotus} spp. is overlooked. Major wetlands are priority habitats for bats conservation in the Reatini Mountain: they are selected significantly from all species, with significant difference between the frequency of use and availability.} %% remember to close the abstract text block brace!!