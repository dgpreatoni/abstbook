% Abstract file structure example : 
% \abstitle{title here}
% \absauthors{names and superscripts for affiliations here}
% \absaddress{affiliations, starting each one with its superscripts, separate affiliations with a \break}
% \abstext{
% \index{author abbreviated name, to be placed in authors' index}
% \index{create an index entry for each author}
%  The abstract text
% }

%% Abstract title
\abstitle{The buzz of drinking on the wing in echolocating bats}

%% Author names
\absauthors{D. \textsc{Russo}$^{1,2}$, L. \textsc{Ancillotto}$^1$, L. \textsc{Cistrone}$^3$, C. \textsc{Korine}$^4$}

\absaddress{$^1$Wildlife Research Unit, Dipartimento di Agraria, Università degli Studi di Napoli Federico II, Portici, Napoli, Italy\break
$^2$School of Biological Sciences, University of Bristol, Bristol, United Kingdom\break
$^3$Forestry and Conservation, Cassino, Frosinone, Italy\break
$^4$Mitrani Department of Desert Ecology, Swiss Institute for Dryland Environmental and Energy Research, Jacob Blaustein Institutes for Desert Research, Ben-Gurion University of the Negev, Midreshet Ben-Gurion, Israel}

%% Abstract text
\abstext{
%% Author names for index. State each author separately using \index{Doe J.}
\index{Russo D.}
\index{Ancillotto L.}
\index{Cistrone L.}
\index{Korine C.}
%% The actual abstract text goes here
Bats are known to broadcast rapid sequences of echolocation calls, named ``drinking buzzes'', when they approach water to drink on the wing. So far this phenomenon has received little attention.

We recorded echolocation sequences of drinking bats for 14 species, for 11 of which we also recorded feeding buzzes, and demonstrated that drinking buzzes are common among low duty cycle echolocators, but absent in high duty cycle echolocators such as rhinolophoids, probably in relation to the fact that the latter perform nasal echolocation so they can spread the mouth to drink while calling.

We also show that drinking buzzes are structurally different from feeding buzzes. Based on the different sensorial tasks faced by feeding and drinking bats, we hypothesize that the drinking buzz structure will differ from that of feeding buzzes, since unlike the latter drinking buzzes are not designed to detect and track mobile prey. We show that the buzz II phase common in feeding buzzes of many bat species is absent in drinking buzzes, i.e. call frequency is not lowered to broaden the sonar beam, since the task does not imply tracking fast moving prey. Pulse rate in drinking buzzes is also lower than in feeding buzzes, as predicted since the high pulse rate typical of feeding buzzes is important to update rapidly the relative location of moving targets.} %% remember to close the abstract text block brace!!