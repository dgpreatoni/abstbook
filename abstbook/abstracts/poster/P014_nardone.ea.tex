% Abstract file structure example : 
% \abstitle{title here}
% \absauthors{names and superscripts for affiliations here}
% \absaddress{affiliations, starting each one with its superscripts, separate affiliations with a \break}
% \abstext{
% \index{author abbreviated name, to be placed in authors' index}
% \index{create an index entry for each author}
%  The abstract text
% }

%% Abstract title
\abstitle{The social calls of \emph{Hypsugo savii}, a potential context-dependent repertoire}

%% Author names
\absauthors{V. \textsc{Nardone}$^1$, L. \textsc{Ancillotto}$^1$, D. \textsc{Russo}$^{1,2}$}

\absaddress{$^1$Wildlife Research Unit, Laboratorio di Ecologia Applicata, Dipartimento di Agraria. Università degli Studi di Napoli Federico II. Via Università 100, 80055 Portici (NA), Italy\break
$^2$School of Biological Sciences, University of Bristol, Bristol, UK}

%% Abstract text
\abstext{
%% Author names for index. State each author separately using \index{Doe J.}
\index{Nardone V.}
\index{Ancillotto L.}
\index{Russo D.}
%% The actual abstract text goes here
Bats emit a variety of calls when engaging in social activities, both agonistic and affiliative (e.g. courting calls). The study and description of bats’ social calls has been important in uncovering the cryptic aspects of bats’ diversity, biogeography and behaviour, particularly in the case of European pipistrelles.

Here we report and describe the diversity of social calls emitted by a common Mediterranean species, the Savi’s pipistrelle \emph{Hypsugo savii}. By recording bat activity in different contexts (e.g. at foraging, drinking and roosting sites) and locations of central and southern Italy, and at different times of the year, corresponding to different life stages, we identified a repertoire of at least 3 different types of social calls, describing their structure and proposing potential different functions for each of them.

We found two types of single-component calls: type one is characterized by a frequency-modulated quasi-constant-frequency structure (FM-QCF), type two has a frequency-modulated initial portion followed by a swing. These two single-component types of social calls are emitted in succession and may be repeated many times in multi-component social calls. The latter were highly variable in their structure and repetition rate. Multi-component social calls were recorded more frequently in late summer and early autumn at drinking sites.
} %% remember to close the abstract text block brace!!