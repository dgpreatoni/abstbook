% Abstract file structure example : 
% \abstitle{title here}
% \absauthors{names and superscripts for affiliations here}
% \absaddress{affiliations, starting each one with its superscripts, separate affiliations with a \break}
% \abstext{
% \index{author abbreviated name, to be placed in authors' index}
% \index{create an index entry for each author}
%  The abstract text
% }

%% Abstract title
\abstitle{Sulle tracce del vespertilio di Brandt: primi dati sulla scelta degli habitat in area alpina}

%% Author names
\absauthors{R. \textsc{Toffoli}, P. \textsc{Culasso}}

\absaddress{Chirosphera Associazione per lo studio e la tutela dei chirotteri e l’ambiente. email: \url{chirosphera@gmail.com}}

%% Abstract text
\abstext{
%% Author names for index. State each author separately using \index{Doe J.}
\index{Toffoli R.}
\index{Culasso P.}
%% The actual abstract text goes here
Tra luglio e settembre 2010 nel Parco Naturale Alpe Veglia e Alpe Devero e relativa ZPS IT1140016 è stato condotto uno studio sulla selezione degli habitat di caccia e sulla scelta dei siti di rifugio da parte di \emph{Myotis brandtii} con la metodologia del \textit{radiotracking}.  L’indagine si è svolta nelle località Piana di Devero (1640 m s.l.m. circa) e Piana di Veglia (1720 m s.l.m.). Sono stati dotati di radiotrasmittente 4 maschi adulti (3 a Devero ed 1 a Veglia) e 2 femmine adulte (a Veglia), identificati sulla base della dentizione, forma del pene e nei casi dubbi geneticamente. Gli animali sono stati seguiti complessivamente per 28 giornate (min 2 max 6 per individuo) totalizzando 340 \textit{fix} (min 10 max 88 per individuo) rilevati da due squadre di operatori che simultaneamente ricavavano la direzione del segnale: successivamente sono state ricavate le localizzazioni degli animali con il \textit{software} ArcView GIS 3.2 e opportune estensioni. 

Si è definita l’area frequentata con il metodo del Minimo Poligono Convesso (MCP), includendo tutti i \textit{fix}, i rifugi e i siti di cattura. L’area utilizzata in fase di attività, esclusi  rifugi e siti di cattura, è stata calcolata con il metodo kernel 95\% sul totale dei \textit{fix} ottenuti, identificando le aree a maggiore frequentazione (\textit{core area}) con un kernel 50\%. Il parametro \textsl{h} è stato determinato con il metodo \textsl{lscv}. Per ogni coppia di individui nella stessa area di studio è stata calcolata la percentuale di sovrapposizione delle aree kernel 95\% e delle \textit{core areas}. La selezione dell’habitat è stata valutata con il \textit{software} Resouce Selection considerando come disponibili le tipologie ambientali all’interno dell’area kernel 95\% e la loro distribuzione percentuale e come utilizzate le tipologie ambientali in cui ricadeva ogni singola localizzazione, e quindi la percentuale di \textit{fix} in ciascuna di esse, a due livelli di dettaglio, corrispondenti all’area kernel 95\% e 50\%.

Le estensioni delle aree frequentate hanno mostrato i seguenti \textit{range} tra gli individui: MCP 79.4 ha min -- 165.0 ha max; kernel 95\% 76.4 ha min -- 113.7 ha max; kernel 50\% 11.3 ha min -- 15.9 ha max. Il grado di sovrapposizione delle aree kernel 95\% tra coppie di individui mostra valori variabili tra il 46 e l’84\% e valori minori per le aree kernel 50\%, compresi tra il 18 e il 65\%. Risultano selezionate negativamente, seppur siano ben rappresentate nell’area di studio e ampiamente frequentate dagli individui seguiti, le aree a lariceta e cembreta in particolare a Veglia (p<0.0001), mentre risultano selezionati positivamente i prati-pascoli in entrambe le località (p<0.001), con particolare evidenza per le aree kernel 50\%.

È stata effettuata una descrizione delle \textit{core area} tramite rilievi sul campo evidenziando come lo stato erboso sia sempre rappresentato e, come siano sempre presenti, ma in proporzioni decrescenti, lo strato arboreo e quallo arbustivo. Non sembra essere significativa l’altezza del manto erboso.  La presenza di esemplari arborei cavati, maturi, senescenti, deperienti con diametro del fusto di almeno 30 cm è stata rilevata nell’80.9\% dei punti di rilievo. La distanza media tra gli esemplari arborei è nell’81.8\% dei casi ricadente nell’intervallo 5--10 metri o superiore, caratteristico di un popolamento non serrato, ma con ampi spazi aperti. La complessità ambientale nelle \textit{core area} è elevata, essendo prevalente la presenza contemporanea dei 3 strati vegetazionali, oltre alle aree caratterizzate interamente da prato-pascoli. A questa si aggiunge la presenza di habitat acquatici (torbiere, torrenti, laghi) che si collocano entro distanze ravvicinate alle \textit{core area}. La presenza di edificati  era marginale e le fonti luminose raramente presenti.

In totale sono stati individuati 12 rifugi utilizzati dagli individui radio marcati. Di questi il 16.6\% era in esemplari arborei mentre il restante in edifici, localizzandosi prevalentemente in intercapedini dei tetti.

In conclusione, nell’area indagata la complessità ambientale sembra avere un ruolo chiave per \emph{Myotis brandtii} poiché assicura una disponibilità trofica adeguata. Non è evidente una stretta dipendenza dalla risorsa forestale, a differenza di altre aree dove è stata descritta l’ecologia della specie. 
} %% remember to close the abstract text block brace!!