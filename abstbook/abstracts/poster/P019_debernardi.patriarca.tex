% Abstract file structure example : 
% \abstitle{title here}
% \absauthors{names and superscripts for affiliations here}
% \absaddress{affiliations, starting each one with its superscripts, separate affiliations with a \break}
% \abstext{
% \index{author abbreviated name, to be placed in authors' index}
% \index{create an index entry for each author}
%  The abstract text
% }

%% Abstract title
\abstitle{Monitoraggio, tutela e valorizzazione di una colonia di \emph{Myotis myotis} e \emph{Myotis blythii}: un caso di studio a lungo termine basato su tecniche non invasive}

%% Author names
\absauthors{P. \textsc{Debernardi}, E. \textsc{Patriarca}}

\absaddress{S.Te.P., c/o Museo Civ. St. Naturale, c.p. 89, 10022 Carmagnola (TO). email: \url{teriologi@gmail.com}}

%% Abstract text
\abstext{
%% Author names for index. State each author separately using \index{Doe J.}
\index{Debernardi P.}
\index{Patriarca E.}
%% The actual abstract text goes here
La colonia di \emph{Myotis myotis} e \emph{M. blythii} dell’Abbazia di Staffarda (Revello, CN) è stata oggetto di indagini basate su tecniche non invasive, con i benefici e i limiti che ciò comporta. 

Utilizza un \textit{roost} posto a piano campagna come \textit{nursery} e, da fine agosto, per l’accoppiamento. Il sito ha microclima simile a quello descritto per grotte calde di latitudine inferiore (nel 2013--2014, da metà aprile a metà ottobre: umidità relativa costantemente prossima al 100\%, escursione termica giornaliera prevalentemente < 1~\degree{}C, temperatura media 18.5~\degree{}C), discostandosi per una maggior escursione termica stagionale (15.4~\degree{}C).  I primi esemplari arrivano fra il 29 marzo e il 15 aprile (mediana 5 aprile, dati di 13 anni); la dispersione termina in novembre, benché sporadicamente si possano osservare esemplari anche in pieno inverno. 

Dati di consistenza sono stati raccolti irregolarmente fra il 1993 e il 2003 e, nei 12 anni successivi, in modo standardizzato, effettuando 2--3 censimenti/anno basati su videoriprese della sciamatura serale e successivo conteggio degli eventuali individui rimasti nel \textit{roost}. Nel secondo periodo sono stati rilevati fra 1034 e 1402 esemplari di età $\geq$1 anno, con variazioni talora notevoli in anni successivi. Alla luce dei dati meteorologici, viene discussa la possibilità che le medesime siano dovute a trasferimenti temporanei di esemplari (maschi, femmine non gravide/non allattanti) con esigenze di termoregolazione diverse da quelle delle femmine gravide/allattanti; la verifica dell’ipotesi richiederebbe ripetute operazioni di cattura, implicanti forte disturbo e perciò evitate.
 
L’età dei piccoli, stimata dall’aspetto e misurando l’avambraccio degli esemplari fotografati nel \textit{roost} nottetempo in presenza di riferimenti metrici, ha consentito di collocare i parti più precoci (verificatesi in 5 anni nella prima settimana di giugno, in 4 anni nell’ultima di maggio e in 3 nella seconda di giugno) e individuare il periodo di maggior frequenza dei parti (in 10 anni nelle prime due settimane di giugno, in 2 anni più tardivamente); le relative date risultano correlate con le temperature medie di aprile-maggio.

Nella prima metà del luglio 2013 sono state condotte tre sessioni di rilevamento bioacustico per individuare le direttrici di spostamento degli esemplari dopo l’uscita serale dall’abbazia, al fine di tenerne conto nella progettazione di interventi di miglioramento ambientale finanziati attraverso il PSR (realizzazione di siepi/filari arborei e zone umide). In ogni sessione sono stati analogamente monitorati sei potenziali punti di transito, disposti a raggiera intorno all’abbazia, per complessive 36 ore di rilevamento. Sono state registrate 2523 sequenze di ecolocalizzazione, dalla cui analisi si è ricavata una caratterizzazione preliminare della chirotterofauna dell’area. Le sequenze attribuite al genere \emph{Myotis} (587, ripartite omogeneamente nelle 3 sessioni) suggeriscono modalità di dispersione degli esemplari della colonia condizionate in maniera opposta dalla presenza/assenza di filari arborei e di sorgenti luminose artificiali, e che espongono a rischio di mortalità per attraversamento di una strada a traffico intenso. Conseguentemente, si è suggerita un’ubicazione degli interventi di miglioramento ambientale volta ad agevolare gli spostamenti e ridurre il rischio di mortalità.

Nei 25 anni decorsi dalla ``scoperta'' della colonia sono stati realizzati interventi gestionali all’interno dell’abbazia comprendenti: esclusione dell’accessibilità al \textit{roost} per il pubblico; modificazione dell’accesso dei chirotteri per consentirne il transito diretto fra il \textit{roost} e l’esterno (senza attraversare volumi interni com’era in precedenza); raccolta e periodica rimozione del guano; disattivazione di un impianto di illuminazione decorativa dell’area antistante il \textit{roost}; realizzazione di un circuito con tre telecamere per consentire ai visitatori dell’abbazia di osservare i pipistrelli senza disturbarli; collocazione di una \textit{webcam} davanti all’accesso del \textit{roost}, per l’osservazione della sciamatura serale e del rientro all’alba degli esemplari attraverso Internet. 
} %% remember to close the abstract text block brace!!