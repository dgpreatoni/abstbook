% Abstract file structure example : 
% \abstitle{title here}
% \absauthors{names and superscripts for affiliations here}
% \absaddress{affiliations, starting each one with its superscripts, separate affiliations with a \break}
% \abstext{
% \index{author abbreviated name, to be placed in authors' index}
% \index{create an index entry for each author}
%  The abstract text
% }

%% Abstract title
\abstitle{The bat fauna of four cavities in south-west Sicily: microclimatic analysis and phenology of communities}

%% Author names
\absauthors{A. \textsc{Fulco}$^{1,3}$, M. \textsc{Vattano}$^{2,3}$, P. \textsc{Valenti}$^{1,3}$, G. \textsc{Madonia}$^2$, M. \textsc{Lo Valvo}$^1$}

\absaddress{$^1$Dipartimento di Scienze e Tecnologie Biologiche, Chimiche e Farmaceutiche, Laboratorio di Zoologia applicata, Università degli Studi di Palermo, Via Archirafi 18, I-90123 Palermo, Italy\break
$^2$Dipartimento di Scienze della Terra e del Mare, Università degli Studi di Palermo, via Archirafi 22, 90123 Palermo, Italy; email: \url{marco.vattano@unipa.it}, \url{giuliana.madonia@unipa.it}\break
$^3$Associazione Naturalistica Speleologica ``Le Taddarite'', via Terrasanta 46, 90141 Palermo, Italy}

%% Abstract text
\abstext{
%% Author names for index. State each author separately using \index{Doe J.}
\index{Fulco A.}
\index{Vattano M.}
\index{Valenti P.}
\index{Madonia G.}
\index{Lo Valvo M.}
%% The actual abstract text goes here
Caves are elective shelters for bat fauna, above all from a climatic point of view. The ``buffer effect'' on the variability of environmental parameters of cavities, make them a suitable habitat for bats. The choice of roosting sites, the shift of colonies from one chamber or passage to another and the different species composition in the communities during the year, might be linked to changes in the microclimatic parameters in the cavities. In order to explain the real links between the roosts climate and the cave bats communities dynamics, a monitoring protocol both environmental and faunal, has been applied on four natural cavities in south-west Sicily (Grotta del Salnitro, Grotta dell’Acqua Fitusa, Grotta dei Personaggi, Grotta Barone). Three of these cavities are home to large bat colonies, while the fourth cavity is not used and serves as a control. Inside these caves 60 dataloggers (T/Rh) have been installed and periodical inspections and captures have been carried out in order to collect data on bats.

Inside the three caves seven bat species were recorded: \emph{Rhinolophus euryale}, \emph{R. ferrumequinum}, \emph{R. hipposideros}, \emph{Myotis myotis}, \emph{M. capaccinii}, \emph{Pipistrellus kuhlii}, \emph{Miniopterus schreibersii}. In particular, the Grotta del Salnitro is home to \emph{M. myotis}, \emph{M. capaccinii}, \emph{M. schreibersii} and only occasionally to \emph{R. euryale}, \emph{P. kuhlii}, the Grotta dell'Acqua Fitusa is occupied by a community of \emph{R. euryale}, \emph{R. ferrumequinum}, \emph{R. hipposideros}, \emph{M. myotis}, \emph{M. capaccinii}, \emph{M. schreibersii}, while the Grotta dei Personaggi hosts only a community of \emph{R. euryale}. Both the species composition and the environments used by the three communities vary during the year.

The research areas are currently being monitored and, according to the first results, we assume a different use of the microenvironments, depending on temperature variations, and a different phenology compared to the known data on peninsular Italy, probably because of the significant latitudinal and climatic difference of Sicily.
} %% remember to close the abstract text block brace!!