
% \abstitle{title here}
% \absauthors{names and superscripts for affiliations here}
% \absaddress{affiliations, starting each one with its superscripts, separate affiliations with a \break}
% \abstext{
% \index{author abbreviated name, to be placed in authors' index}
% \index{create an index entry for each author}
%  The abstract text
% }

%% Abstract title
\abstitle{I chirotteri delle Riserve Naturali Statali di Siena: conoscenza e conservazione}

%% Author names
\absauthors{S. \textsc{Vergari}$^1$, G. \textsc{Dondini}$^1$, C. \textsc{Saveri}$^2$}

\absaddress{$^1$Centro Naturalistico e Archeologico dell’Appennino Pistoiese, Via L. Orlando 100, Campo Tizzoro (Pistoia)\break
$^2$Corpo Forestale dello Stato, Ufficio Territoriale per la Biodiversità Siena, via Cassia Nord 7, 53100 Siena}

%% Abstract text
\abstext{
%% Author names for index. State each author separately using \index{Doe J.}
\index{Dondini G.}
\index{Vergari S.}
\index{Saveri S.}
%% The actual abstract text goes here
Nel 2014 è iniziato un progetto di studio dei chirotteri delle Riserve Statali Naturali di Cornocchia e di Tocchi (Siena), entrambe gestite dal Corpo Forestale dello Stato.

L’obiettivo principale è quello di acquisire dati sulle varie specie presenti, sui \textit{roost} e sulle principali aree di foraggiamento, al fine di inserire indicazioni nei piani di gestione delle Riserve stesse.

La grande varietà di comportamenti presentata da questo ordine di Mammiferi impone l’adozione di metodologie di indagine diversificate ed articolate, così da poter rilevare tutte le specie presumibilmente presenti nell’area di studio. Si è quindi proceduto a visitare, durante il giorno, tutti i potenziali rifugi come gli edifici abbandonati. Durante la notte si sono effettuati rilievi con \textit{bat detector} Pettersson D-240X, registrando in digitale con registratore Edirol R-09. I sonogrammi sono stati analizzati con \textit{BatSound} 3.10. 

Il lavoro è stato condotto nelle due Riserve Statali Naturali di Cornocchia e di Tocchi, attraverso una serie di transetti percorsi in auto ad una velocità di circa 10 km/h, registrando tutti i contatti ultrasonori, con transetti percorsi a piedi e attraverso il rilievo da punti fissi, sostando 15 minuti e registrando tutti i passaggi. Per la valutazione delle aree di foraggiamento sono state definite le seguenti categorie: (1) querceto, (2) prati pascoli, (3) aree umide, (4) aree antropizzate. In fase di elaborazione, per avere dei risultati confrontabili sono stati considerati solo i dati da punti di ascolto. Per ogni tipologia ambientale sono stati eseguiti 3 punti di ascolto da 15 minuti ciascuno. Sono state identificate 10 specie per la Riserva Naturale Statale di Cornocchia, tra le quali si annoverano \emph{Rhinolophus hipposideros}, \emph{Rhinolophus ferrumequinum} e \emph{Myotis emarginatus} e un’importante colonia di \emph{Plecotus austriacus}, la prima osservata nella provincia di Siena. Per la Riserva Naturale Statale di Tocchi le specie identificate sono 5, con un interessante dato sulla presenza di \emph{Myotis daubentonii}, che utilizza il fiume Merse come area di foraggiamento.
} %% remember to close the abstract text block brace!!