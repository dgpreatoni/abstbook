% Abstract file structure example : 
% \abstitle{title here}
% \absauthors{names and superscripts for affiliations here}
% \absaddress{affiliations, starting each one with its superscripts, separate affiliations with a \break}
% \abstext{
% \index{author abbreviated name, to be placed in authors' index}
% \index{create an index entry for each author}
%  The abstract text
% }

%% Abstract title
\abstitle{Geographical distribution of the bat fauna of Sicily: current state of knowledge}

%% Author names
\absauthors{A. \textsc{Fulco}, M. \textsc{Lo Valvo}}

\absaddress{Dipartimento di Scienze e Tecnologie Biologiche, Chimiche e Farmaceutiche, Laboratorio di Zoologia applicata, Università degli Studi di Palermo, Via Archirafi 18, I-90123 Palermo, Italy}

%% Abstract text
\abstext{
%% Author names for index. State each author separately using \index{Doe J.}
\index{Fulco A.}
\index{Lo Valvo M.}
%% The actual abstract text goes here
Sicily is the widest region in Italy and also the largest island in the Mediterranean sea. In spite of that, data about the Sicilian bat fauna are scarse and fragmentary, above all as regards its geographical distribution, and still widely inadequate if compared to the richness of habitats and the great biogeographical value of this area. Since the past few years we have carried out a cognitive survey for the achievement of a Sicilian bat fauna atlas and the guidelines on the conservation of species and the sustainable use of habitats. 

The survey develops into different stages: first of all an accurate bibliographic research to get all previous data and the consultation of the most important zoological collections. The following step is based on a field survey with the main aim of filling the gap of knowledge in some areas of the region where no occurences have been recorded. In this stage data have been collected through inspections in natural or artificial shelters both known and/or potential (with a special attention on karstic cavities), captures (by means of mist net, harp trap, hand nets) and bioacoustic sampling (bat detector Petterson D1000X). The last stage, still in progress, consists in the analysis of the data collected and processing, together with past data, of the distribution maps.

All data obtained so far agree with the expected data based on the ecological features of the species. The finding of new colonies during the exploration of various hypogeal sites and large regional areas, so far little or not at all known, allowed us to update the checklist of the sicilian bat fauna and build up preliminary distribution maps. In the current state of knowledge on the Sicily region territory the occurrence of 24 species has been recorded: \emph{Rhinolophus euryale}, \emph{Rhinolophus ferrumequinum}, \emph{Rhinolophus hipposideros}, \emph{Rhinolophus mehelyi}, \emph{Myotis bechsteinii}, \emph{Myotis blythii}, \emph{Myotis capaccinii}, \emph{Myotis daubentonii}, \emph{Myotis emarginatus}, \emph{Myotis myotis}, \emph{Myotis mystacinus}, \emph{Myotis nattereri}, \emph{Myotis punicus}, \emph{Pipistrellus kuhlii}, \emph{Pipistrellus pipistrellus}, \emph{Pipistrellus pygmaeus}, \emph{Nyctalus lasiopterus}, \emph{Hypsugo savii}, \emph{Eptesicus serotinus}, \emph{Barbastella barbastellus}, \emph{Plecotus auritus}, \emph{Plecotus austriacus}, \emph{Miniopterus schreibersii}, \emph{Tadarida teniotis}. The most frequently recorded species, occurred in all nine sicilian provinces, are: \emph{P. kuhlii}, \emph{P. pipistrellus}, \emph{M. schreibersii} and \emph{T. teniotis}. 

For the moment the distribution maps we have done are not fully exhaustive for such a wide area, though they represent an important synthesis of the current knowledge and a good starting point for future studies. We believe that further researches, particularly carried in the woodland and on the Sicily minor island, might enhance both the checklist and echological knowledge about those species which are almost totally absent in Sicily.
} %% remember to close the abstract text block brace!!