
% \abstitle{title here}
% \absauthors{names and superscripts for affiliations here}
% \absaddress{affiliations, starting each one with its superscripts, separate affiliations with a \break}
% \abstext{
% \index{author abbreviated name, to be placed in authors' index}
% \index{create an index entry for each author}
%  The abstract text
% }

%% Abstract title
\abstitle{Rare but in healty refuge: low heavy metal accumulation in \emph{Pipistrellus hanaki}}

%% Author names
\absauthors{D. \textsc{Scaravelli}$^1$, P. \textsc{Georgiakakis}$^2$, L. \textsc{Filippini}$^1$, A. \textsc{Zaccaroni}$^1$}

\absaddress{$^1$Dipartimento di Scienze Mediche Veterinarie, Università di Bologna, via Tolara di sopra 50, Ozzano Emilia. email: \url{dino.scaravelli@unibo.it}\break
$^2$Natural History Museum of Crete, University of Greece, P.O. Box 2208, GR–71 409 Irakleion, Greece. email: \url{pangeos@nhmc.uoc.gr}}

%% Abstract text
\abstext{
%% Author names for index. State each author separately using \index{Doe J.}
\index{Scaravelli D.}
\index{Georgiakakis P.}
\index{Filippini L.}
\index{Zaccaroni A.}
%% The actual abstract text goes here
Among bats there are many species considered at risk for their conservation. Pollution can be a strong pressure on different species and, among others pollutants, high concentration of heavy metals can produce both acute and chronic effects, rendering bats suitable for use as an indicator of general environmental conditions.

\emph{Pipistrellus hanaki} Benda \& Hulva, 2004 is one of the rarest species of bats in Europe. The species is confined in Crete with an endemic subspecies and is present only in Cyrenaica (North Libya). This tiny species is supposed to feed on small insects, and a first attempt to evaluate the level of pollutants was done during a study on its habitat ecology. Feces were collected from roosts close to Aravanes oak forest and Margarites village. In Margarites the landscape is characterized by an agricultural mosaic where old and new agricultural practices are mixed, with large extensions of Mediterranean shrub vegetation, oak stands and small creeks partially dry.

In the laboratory feces were microwave wet digested and later analyzed by Inductively Coupled Plasma Atomic Emission Spectroscopy (ICP-AES). The following metals were searched: Al, As, Cd, Cr, Cu, Fe, Mn, Ni, Pb, Se, Zn, Hg.

In Aravanes the heavy metals concentrations found (in mg/kg) were the following: Al 214.593, As 0.154, Cd 0.900, Cr 2.670, Cu 59.619, Fe 343.449, Mn 22.039, Ni 15.911, Pb 4.040, Se 0.946, Zn 79.922, Hg 9.859, whereas for Margarites were:  Al 322.596, As 0.062, Cd 0.131, Cr 2.627, Cu 66.407, Fe 341.817, Mn 53.535, Ni 13.656, Pb 1.852, Se 0.517, Zn 99.368, Hg 0.315.

Metal concentrations are very low in all samples of \emph{P. hanaki} feces and just Cd, Cr and Pb are present at high voncentrations, below the mean values reported in other similar studies on bats. The only significantly high values were found for Ni and Hg in Aravanes, quite high if compared with studies in other \emph{Pipistrellus} species, but probably related to specific prey items that have contributed to accumulation. In the close future, other samples will be analyzed in order to understand the exact diet composition and to assess the mechanism of the occurring biomagnification process.

At the moment there are no doubts that \emph{P. hanaki} lives in a not polluted environment and that the main conservation problem is related to possible change in land use.
} %% remember to close the abstract text block brace!!