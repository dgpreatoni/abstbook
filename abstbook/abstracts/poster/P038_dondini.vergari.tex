
% \abstitle{title here}
% \absauthors{names and superscripts for affiliations here}
% \absaddress{affiliations, starting each one with its superscripts, separate affiliations with a \break}
% \abstext{
% \index{author abbreviated name, to be placed in authors' index}
% \index{create an index entry for each author}
%  The abstract text
% }

%% Abstract title
\abstitle{Chirotteri di tre aree forestali costiere toscane: Riserva Naturale Statale di Cecina, Oasi WWF Bosco di Cornacchiaia e Oasi WWF Dune di Tirrenia}

%% Author names
\absauthors{G. \textsc{Dondini}, S. \textsc{Vergari}}

\absaddress{Centro Naturalistico e Archeologico dell’Appennino Pistoiese, Via L. Orlando 100, Campo Tizzoro (Pistoia)}

%% Abstract text
\abstext{
%% Author names for index. State each author separately using \index{Doe J.}
\index{Dondini G.}
\index{Vergari S.}
%% The actual abstract text goes here
Nel periodo 2011--2014 sono stati compiuti approfondimenti sulla struttura della chirotterofauna in tre aree forestali costiere toscane: la Riserva Naturale Statale di Cecina, gestita dal Corpo Forestale dello Stato e le Oasi WWF Bosco di Cornacchiaia e Dune di Tirrenia (gestite dal WWF di Pisa). Questo lavoro ha permesso di indagare tipologie forestali costiere e di compiere confronti sulla ricchezza specifica in queste tre diverse aree. Le metodologie di indagine impegnate per poter rilevare tutte le specie presumibilmente presenti sono state l’ispezione diretta di tutti i potenziali rifugi, e durante la notte registrazioni con \textit{bat detector} Pettersson D-240X e successivamente analizzate con \textit{BatSound} 3.10.

La Riserva di Cecina è caratterizzata da una vegetazione forestale dominata dal pino marittimo, pino d'Aleppo e pino domestico. L’oasi di Tirrenia è invece dominata dal pino marittimo, mentre Cornacchiaia si presenta come l’area forestale più diversificata e caratterizzata dalla presenza di un bosco planiziale, composto da pini domestici secolari, leccio, frassino ossifillo, ontano nero, pioppo bianco e farnia. Le specie rilevate sono 7 per la Riserva Naturale di Cecina, per la quale non sono stati trovati \textit{roost} di colonie riproduttive/svernamento; 4 specie per l’Oasi di Tirrenia, senza ritrovamenti di rifugi di colonie riproduttive/svernamento; 8 specie per l’Oasi di Cornacchiaia, nella quale sono state rilevate due colonie riproduttive, rispettivamente di \emph{Rhinolophus ferrumequinum} e \emph{Myotis emarginatus}. In conclusione i complessi forestali costieri, in particolare quelli caratterizzati da un certo grado di naturalità, rappresentano importanti aree di foraggiamento e di rifugio per numerose specie di pipistrelli.
} %% remember to close the abstract text block brace!!