
% \abstitle{title here}
% \absauthors{names and superscripts for affiliations here}
% \absaddress{affiliations, starting each one with its superscripts, separate affiliations with a \break}
% \abstext{
% \index{author abbreviated name, to be placed in authors' index}
% \index{create an index entry for each author}
%  The abstract text
% }

%% Abstract title
\abstitle{Radure intrasilvatiche e attività di foraggiamento dei Chirotteri: interventi nell’ambito del progetto LIFE ``Save the flyers''}

%% Author names
\absauthors{G. \textsc{Dondini}$^1$, S. \textsc{Vergari}$^1$, G. \textsc{Ceccolini}$^2$, A. \textsc{Cenerini}$^2$}

\absaddress{$^1$Centro Naturalistico e Archeologico dell’Appennino Pistoiese, Via L. Orlando 100, 51028 Campo Tizzoro (Pistoia)\break
$^2$2Associazione CERM Centro Rapaci Minacciati, Via Santa Cristina 6, 58055 Rocchette di Fazio (Grosseto)}

%% Abstract text
\abstext{
%% Author names for index. State each author separately using \index{Doe J.}
\index{Dondini G.}
\index{Vergari S.}
\index{Ceccolini G.}
\index{Cenerini A.}
%% The actual abstract text goes here
Le radure all’interno delle foreste rappresentano importanti serbatoi di biodiversità. Per i chirotteri questi particolari microhabitat costituiscono utili aree di foraggiamento e possibili vie per trovare rifugi in alberi cavi. L’impiego della tecnica GLA (\textit{Gap Light Analyzer}) consente di ``quantificare'' la struttura della canopea, parametro di fondamentale importanza per la valutazione dell’attività di foraggiamento e di utilizzo da parte della chirotterofauna.

Il protocollo sperimentale prevedeva lo studio di due aree sul Monte Penna e di cinque aree sul Monte Amiata in Toscana. In queste aree sono stati compiuti 4 rilievi con \textit{bat detector} (Pettersson D240X) prima e dopo l'esecuzione di interventi di taglio. Per ogni sessione di rilievo sono stati registrati tutti gli impulsi ultrasonori emessi dal tramonto fino alle 2:00 di notte. I rilievi sono stati effettuati tra maggio e luglio negli anni 2012, 2013 e 2014. Questo ha permesso di evidenziare l’andamento nel tempo relativamente all’utilizzo delle radure da parte della chirotterofauna. Le aperture sono state valutate utilizzando una macchina fotografica Nikon D80 con un obiettivo \textit{fisheye} a 180\degree{} e le foto sono state successivamente analizzate con il software GLA.

I dati raccolti evidenziano una significativa e positiva correlazione tra radure ed attività di foraggiamento. Il confronto tra il numero complessivo di contatti ultrasonici pre e post1 e post2 evidenzia una rilevante differenza d’uso. In particolare la geometria della canopea, ovvero la percentuale e la distribuzione delle aperture nella volta forestale, influenza le attività di foraggiamento. Risulta infatti che il fattore importante non è solo la percentuale di apertura, ma anche come questa è ripartita. I risultati di questo studio rivelano, dunque, come anche piccole radure possano favorire le attività di foraggiamento dei chirotteri.
} %% remember to close the abstract text block brace!!