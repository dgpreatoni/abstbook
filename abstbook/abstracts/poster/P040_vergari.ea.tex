
% \abstitle{title here}
% \absauthors{names and superscripts for affiliations here}
% \absaddress{affiliations, starting each one with its superscripts, separate affiliations with a \break}
% \abstext{
% \index{author abbreviated name, to be placed in authors' index}
% \index{create an index entry for each author}
%  The abstract text
% }

%% Abstract title
\abstitle{Preliminary data on bats of Ankober highland area (Ethiopia)}

%% Author names
\absauthors{S. \textsc{Vergari}$^1$, G. \textsc{Dondini}$^1$, R. \textsc{Barocco}$^2$, M. \textsc{Tarkegen Nigatu}$^3$, A. \textsc{Barili}$^2$, S. \textsc{Gentili}$^2$}

\absaddress{$^1$Centro Naturalistico e Archeologico dell’Appennino Pistoiese, Via L. Orlando 100, Campo Tizzoro (Pistoia)\break
$^2$C.A.M.S. Centro di Ateneo per i Musei Scientifici, Università degli Studi di Perugia\break
$^3$Faculty of Natural and Computational Science and Department of Biology, Woldia University}

%% Abstract text
\abstext{
%% Author names for index. State each author separately using \index{Doe J.}
\index{Dondini G.}
\index{Vergari S.}
\index{Barocco R.}
\index{Tarkegen Nigatu M.}
\index{Barili A.}
\index{Gentili S.}
%% The actual abstract text goes here
Ethiopia has unique fauna and flora. It comprehends several important terrestrial ecoregions, such as the Ethiopian Montane Moorlands, the Ethiopian Montane Grasslands and Woodlands, and the Ethiopian Montane Forests. Ethiopian forest cover has declined to 3.56\% of the total. The annual loss of the highland forest areas of Ethiopia has been estimated at between 150000 and 200000 ha. The present research proposal is aimed to fill a gap on knowledge of bat fauna in the Ankober areas, where montane Afro-tropical fragmented forest, grassland, moorland and wetland ecosystems occur. 

The purposes of field surveys on bats were to collect faunal data. Captures with mist nets were conducted in different natural and cultivated habitats. Individuals were released into the wild as soon as possible, reducing the stress at the minimum. Regarding Microchiroptera, ultrasound calls were recorded with a bat detector (Pettersson D240X). Three areas were chosen at different altitudes: Aliyu Amba (1300 m a.s.l.) characterized by a savanna-like vegetation; Lét Marefià (2400 m a.s.l.) below the summit of the Emmemret mountain, at the edge of an important primary forest and characterized by fragmented agricultural areas; Kundi (3700 m a.s.l.) characterized by an afro-alpine mountain vegetation. New data on the foraging activities, on the species observed and on the influence of altitudinal gradient are provided for a poorly investigated area.
} %% remember to close the abstract text block brace!!