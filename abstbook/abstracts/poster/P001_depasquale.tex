% Abstract file structure example : 
% \abstitle{title here}
% \absauthors{names and superscripts for affiliations here}
% \absaddress{affiliations, starting each one with its superscripts, separate affiliations with a \break}
% \abstext{
% \index{author abbreviated name, to be placed in authors' index}
% \index{create an index entry for each author}
%  The abstract text
% }

%% Abstract title
\abstitle{I Chirotteri del Parco Nazionale dell'Appennino Lucano Val d'Agri Lagonegrese}

%% Author names
\absauthors{P.P. \textsc{De Pasquale}}

\absaddress{Wildlife Consulting, Viale S. Mercadante 26, 70132 Bari, Italy}

%% Abstract text
\abstext{
%% Author names for index. State each author separately using \index{Doe J.}
\index{De Pasquale P.P.}
%% The actual abstract text goes here
\textbf{Queto lavoro è presente anche in forma di \textit{extended abstract} a pag. \ref{ext:P001}}

Il presente lavoro è stato condotto nel Parco Nazionale dell’Appennino Lucano Val d'Agri Lagonegrese, nell'anno 2012--2013, con lo scopo di compilare una \textit{checklist} delle specie presenti, analizzarne la distribuzione nei diversi habitat attraverso la progettazione di modelli d'idoneità ambientale specie-specifici e di effettuare un censimento preliminare dei rifugi utilizzati. Le metodologie impiegate hanno incluso l'impiego di rilevatori ultrasonori in espansione temporale, di reti del tipo \textit{mist net} per chirotteri e su 3 individui è stata effettuata una biopsia della pelle per la successiva analisi molecolare dei \textit{taxa} criptici.
Il campionamento bioacustico è stato stratificato rispetto alla disponibilità ambientale, mentre i punti d'ascolto sono stati selezionati in modo random all’interno di ciascuna categoria ambientale.
I modelli sono stati elaborati mediante procedure GIS consultando differenti basi cartografiche, tra cui la cartografia CORINE Land Cover e attraverso l'analisi dei dati raccolti sul campo e delle caratteristiche autoecologiche di ogni singola specie rilevata. Queste caratteristiche sono state successivamente correlate con variabili ambientali generali che possono influenzare la presenza delle specie nel territorio oggetto di studio.   
La progettazione ha previsto la restituzione di cartografie che rappresentano la distribuzione potenziale di ogni specie nell’area di studio, nelle quali il diverso grado di idoneità ambientale è stato suddiviso in 4 categorie. 
Sono stati individuati 12 \textit{roost} utilizzati dai chirotteri e, in particolare, due importanti colonie riproduttive di \emph{Rhinolophus euryale} e \emph{Rhinolophus ferrumequinum}.                                                                                                                                                              
Nel territorio del Parco sono state censite ben 21 specie di chirotteri, alcune delle quali risultano rare in tutto il territorio nazionale, tra cui il Vespertilio di Bechstein (\emph{Myotis bechsteinii})  e il Barbastello (\emph{Barbastella barbastellus}).
} %% remember to close the abstract text block brace!!