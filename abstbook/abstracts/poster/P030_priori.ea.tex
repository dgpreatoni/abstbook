% Abstract file structure example : 
% \abstitle{title here}
% \absauthors{names and superscripts for affiliations here}
% \absaddress{affiliations, starting each one with its superscripts, separate affiliations with a \break}
% \abstext{
% \index{author abbreviated name, to be placed in authors' index}
% \index{create an index entry for each author}
%  The abstract text
% }

%% Abstract title
\abstitle{Morfologia esterna di \emph{Cimex pipistrelli} Jenyns, 1839 con note ecologiche}

%% Author names
\absauthors{P. \textsc{Priori}$^1$, M. \textsc{Amadori}$^1$, L. \textsc{Guidi} $^1$, D. \textsc{Scaravelli}$^2$}

\absaddress{$^1$Dipartimento di Scienze della Terra, della Vita e dell’Ambiente. Università degli Studi di Urbino Carlo Bo, Campus Scientifico, loc. Crocicchia. 61029 Urbino. email: \url{pamela.priori@uniurb.it}\break
$^2$Dipartimento di Scienze Mediche Veterinarie, Università di Bologna, via Tolara di sopra 50, Ozzano Emilia email: \url{dino.scaravelli@unibo.it}}

%% Abstract text
\abstext{
%% Author names for index. State each author separately using \index{Doe J.}
\index{Priori P.}
\index{Amadori M.}
\index{Guidi L.}
\index{Scaravelli D.}
%% The actual abstract text goes here
Tra gli ectoparassiti dei Chirotteri presenti in Italia, una specie ancora poco indagata dal punto di vista morfologico ed ecologico è \emph{Cimex pipistrelli} Jenyns, 1839 (Insecta, Rhynchota, Cimicidae). La specie, la cui descrizione originale risale all’800, è stata recentemente ri-descritta e mostra un’ecologia parassitaria particolare e una morfologia specifica.

Nell’ambito di un ampio studio sull’ecologia parassitaria dei chirotteri italiani, diversi esemplari di \emph{C. pipistrelli} sono stati osservati nei loro comportamenti e poi raccolti nella colonia di \emph{Myotis myotis} e \emph{M. blythii} presente nel sottotetto della chiesa di Vezzano (BZ). Alcuni esemplari adulti di \emph{C. pipistrelli}, sia di sesso maschile che femminile, sono stati preparati per l’osservazione al microscopio elettronico a scansione (S.E.M.).

L’osservazione al S.E.M. ha permesso di evidenziare numerosi caratteri anatomici ed in particolare di riconoscere e misurare i caratteri significativi per il differenziamento tra le specie. In \emph{C. pipistrelli} il pronoto presenta due caratteristiche espansioni laterali notevolmente sviluppate e separate dal capo da un profondo incavo e il rapporto tra larghezza e lunghezza si attesta significativamente tra 2.0 e 2.5. Le antenne di \emph{C. pipistrelli} mostrano la classica divisione in quattro segmenti tipica di tutti gli appartenenti al genere \emph{Cimex}. Sulle antenne sono stati osservati cinque tipi di sensilli (S1, S2, S3, S4, S5). Quattro di essi sono stati già descritti in altre specie di \emph{Cimex}, mentre  il tipo S4 è qui descritto per la prima volta. Data la a particolare posizione e morfologia  di questo tipo di sensilli  è possibile ipotizzare una funzione termo-recettrice. Sulle zampe di \emph{C. pipistrelli}, in corrispondenza dell’articolazione tibio-tarsica, sono state descritte particolari strutture a sperone e, in corrispondeza del pretarso, due unghie unciniformi che possono essere interpretate come adattamenti alla vita parassitaria. \emph{C. pipistrelli}, infatti, ha un regime dietetico ematofago ma non vive sempre sul corpo dell'ospite: si insidia in microambienti in prossimità dell’ospite sul quale sale solo per alimentarsi e su cui si aggrappa grazie alle due unghie unciniformi. Le strutture a forma di sperone invece facilitano la sua deambulazione quando non è attaccato all’ospite. La forma particolarmente acuminata dell’organo copulatore e la presenza del seno paragenitale nella femmina confermano che \emph{C. pipistrelli}, come \emph{C. lectularius}, attua un’inseminazione traumatica con penetrazione extragenitale e successivo movimento degli spermatozoi direttamente nell’emocele della femmina.

L’alto grado di infestazione delle colonie di ``grandi'' \emph{Myotis} in Alto Adige è correlato alla struttura stessa dei \textit{roost} dato che il substrato in travi di legno è in grado di ospitare e ``proteggere'' un gran numero di cimici che mostrano, dalla primavera all’estate, un continuo susseguirsi di generazioni. A confronto i \textit{roost} ipogei che le stesse specie di chirotteri hanno nell’ambito mediterraneo, più freddi e umidi, si rivelano praticamente privi di questi ectoparassiti.
} %% remember to close the abstract text block brace!!