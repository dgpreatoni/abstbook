% Abstract file structure example : 
% \abstitle{title here}
% \absauthors{names and superscripts for affiliations here}
% \absaddress{affiliations, starting each one with its superscripts, separate affiliations with a \break}
% \abstext{
% \index{author abbreviated name, to be placed in authors' index}
% \index{create an index entry for each author}
%  The abstract text
% }

%% Abstract title
\abstitle{I Chirotteri della Liguria: stato attuale delle conoscenze}

%% Author names
\absauthors{M. \textsc{Calvini}}

\absaddress{Via Dante Alighieri, 426 – 18038 Sanremo (IM). email: \url{mara.calvini@gmail.com}}

%% Abstract text
\abstext{
%% Author names for index. State each author separately using \index{Doe J.}
\index{Calvini M.}
%% The actual abstract text goes here
Il presente lavoro riassume lo stato delle conoscenze dei Chirotteri della Liguria e fornisce una \textit{check-list} regionale derivante da progetti di monitoraggio regionali, provinciali e da ricerche personali, a partire dal 2000 fino al 2014.

La definizione della presenza attuale nel territorio regionale ha preso in considerazione ricerche bibliografiche estese alla letteratura ``grigia'' ritenute utili al presente lavoro, ovvero associate a una precisa attribuzione tassonomica o ad una documentazione fotografica esauriente e caratterizzate da una precisa ubicazione  temporale e spaziale, una richiesta di dati inediti ad appassionati e speleologi operanti sul territorio regionale previa validazione delle informazioni e indagini di campo. Queste hanno previsto la ricerca e il controllo dei \textit{roost} attraverso il conteggio degli esemplari a vista, da supporto fotografico o riprese video a infrarossi, controllo di \textit{bat-box} installate a seguito di progetti di ricerca, catture degli individui con l’utilizzo di reti fisse tipo \textit{mist net} posizionate principalmente in prossimità di zone umide; recupero di esemplari in difficoltà o rinvenuti morti e rilevamenti ultrasonori con \textit{bat detector} lungo transetti o stazioni fisse con successiva identificazione dei segnali registrati.

Complessivamente per la regione Liguria sono state attualmente rilevate 25 specie di Chirotteri che rappresentano il 73.5\% di quelle note a livello nazionale: \emph{Rhinolophus hipposideros}, \emph{Rhinolophus ferrumequinum}, \emph{Rhinolophus euryale}, \emph{Myotis daubentonii}, \emph{Myotis capaccinii}, \emph{Myotis mystacinus}, \emph{Myotis} cfr. \emph{nattereri}, \emph{Myotis emarginatus}, \emph{Myotis bechsteinii}, \emph{Myotis myotis}, \emph{Myotis blythii}, \emph{Nyctalus noctula}, \emph{Nyctalus leisleri}, \emph{Pipistrellus pipistrellus}, \emph{Pipistrellus pygmaeus}, \emph{Pipistrellus nathusii}, \emph{Pipistrellus kuhlii}, \emph{Hypsugo savii}, \emph{Eptesicus serotinus}, \emph{Barbastella barbastellus}, \emph{Plecotus auritus}, \emph{Plecotus austriacus}, \emph{Plecotus macrobullaris}, \emph{Miniopterus schreibersii} e \emph{Tadarida teniotis}.

Tutti i siti sono stati georeferenziati e cartografati nel sistema UTM-WGS84. Per quanto riguarda la localizzazione dei rifugi di chirotteri complessivamente noti in Liguria è stata focalizzata l’attenzione sulle specie che compaiono nell'allegato II della Direttiva 92/43/CE corrispondenti alle seguenti caratteristiche: \textit{roost} rispondenti ai criteri di selezione dei siti chirotterologici di particolare interesse conservazionistico proposti a livello nazionale dal GIRC e adattatati alla situazione regionale: \textit{roost} riproduttivi delle specie incluse nell'allegato II della Direttiva 92/43/CE, \textit{roost} di svernamento che ospitano almeno 10 esemplari appartenenti a specie incluse nell'allegato II della Direttiva 92/43/CE. Sono stati individuati 21 siti riproduttivi, di cui 11 di importanza nazionale, appartenenti a \emph{R. hipposideros}, \emph{M. blythii}, \emph{M. emarginatus}, \emph{M. schreibersii}, mentre 35 sono i siti di ibernazione appartenenti a \emph{R. ferrumequinum}, \emph{R. hipposideros}, \emph{B. barbastellus}, \emph{M. capaccinii}.
} %% remember to close the abstract text block brace!!