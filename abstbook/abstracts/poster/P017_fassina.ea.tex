% Abstract file structure example : 
% \abstitle{title here}
% \absauthors{names and superscripts for affiliations here}
% \absaddress{affiliations, starting each one with its superscripts, separate affiliations with a \break}
% \abstext{
% \index{author abbreviated name, to be placed in authors' index}
% \index{create an index entry for each author}
%  The abstract text
% }

%% Abstract title
\abstitle{Una nuova colonia di chirotteri presso Forte San Briccio -– Verona}

%% Author names
\absauthors{C. \textsc{Fassina}$^1$, R. \textsc{Favatà}$^2$, E. \textsc{Moscardo}$^3$, G. \textsc{Piras}$^1$}

\absaddress{$^1$Via Capitello 86/A, 35136 Padova\break
$^2$c/o Associazione All'Ombra del Forte, c/o Forte San Briccio via della Liberazione Lavagno (VR)\break
$^3$c/o Museo Civico di Storia Naturale, Palazzo Lavezola Pompei, Lungadige Porta Vittoria 9, 37129 Verona}

%% Abstract text
\abstext{
%% Author names for index. State each author separately using \index{Doe J.}
\index{Fassina C.}
\index{Favatà R.}
\index{Moscardo E.}
\index{Piras G.}
%% The actual abstract text goes here
Forte San Briccio è una fortificazione del comune di Lavagno (VR) posta a 223 m d'altitudine sulla sommità di una collina che costituisce una delle propaggini più meridionali dei Monti Lessini veronesi. Venne realizzata a partire dal 1883 ed è caratterizzata da una particolare tipologia costruttiva in muratura e laterizio ricoperta da un abbondante terrapieno con funzione difensiva antigranata. All’interno del manufatto, che si sviluppa per oltre 5000 metri quadri su una superficie trapezoidale complessiva di 22000 metri quadrati, sono disponibili alcuni vani con caratteristiche climatiche ed ambientali idonee all’insediamento dei chirotteri. Dopo un’utilizzazione militare ininterrotta, durata fino al 1979, dalla fine del 1900 se n’è tentato il recupero per fini ricreativi e museali.

Nel corso dei sopralluoghi avvenuti nel 2013 per valutarne una nuova utilizzazione a fini ricreativi e museali è stata notata la presenza di chirotteri, ma solo nel corso del 2014 si è potuto accertare la presenza anche in periodo riproduttivo di una cospicua colonia di Rinolofi maggiori. Si è deciso quindi di procedere ad un rilievo chirotterologico che coprisse le più importanti fasi fenologiche durante tutto l'anno. Le indagini hanno appurato come il sito venga utilizzato in tutti i periodi, sebbene dirante lo svernamento siano stati conteggiati solo 25 esemplari di \emph{Rhinolophus ferrumequinum} e 1 \emph{Rhinolophus hipposideros}.

Altre specie presenti sono \emph{Miniopterus schreibersii}, \emph{Myotis myotis}/\emph{blythii}, mediante riscontro visivo, \emph{Myotis emarginatus} con il ritrovamento di un cranio e, con rilevamento bioacustico durante una sessione notturna \emph{Pipistrellus kuhlii}, \emph{Hyspugo savii} ed \emph{Eptesicus serotinus}.

Il sito quindi risulta utilizzato da 5 specie poste in allegato II della direttiva 92/43/CEE e complessivamente da almeno 8 specie di chirotteri. Si è accertata la riproduzione al suo interno di  \emph{Rhinolophus ferrumequinum} ed è utilizzato come \textit{hibernaculum} da \emph{Rhinolophus ferrumequinum} e \emph{Rhinolophus hipposideros}.
} %% remember to close the abstract text block brace!!