
% \abstitle{title here}
% \absauthors{names and superscripts for affiliations here}
% \absaddress{affiliations, starting each one with its superscripts, separate affiliations with a \break}
% \abstext{
% \index{author abbreviated name, to be placed in authors' index}
% \index{create an index entry for each author}
%  The abstract text
% }

%% Abstract title
\abstitle{Monitoraggio dei Chirotteri nel territorio della provincia di Pistoia: risultati e azioni di conservazione}

%% Author names
\absauthors{G. \textsc{Dondini}, S. \textsc{Vergari}}

\absaddress{Itinerari società cooperativa via Forravilla 35, 51028 Pracchia (Pistoia)}

%% Abstract text
\abstext{
%% Author names for index. State each author separately using \index{Doe J.}
\index{Dondini G.}
\index{Vergari S.}
%% The actual abstract text goes here
Il territorio della Provincia di Pistoia è caratterizzato da un’ampia varietà di ambienti, che vanno dalle cime appenniniche alle zone umide della pianura. Il monitoraggio è lo strumento che consente di avere un controllo costante nel tempo sulla dinamica delle zoocenosi, sia in senso spaziale che numerico, e permette di ridurre gli impatti significativi sull’ambiente derivanti dall’attuazione delle opere, dei piani approvati e verifica il raggiungimento degli obiettivi di sostenibilità prefissati. Inoltre consente la stesura di piani per la conservazione delle specie o degli habitat a maggiore rischio. Oltre al censimento delle specie (ed all’acquisizione di informazioni di carattere fenologico), il progetto si prefigge anche di individuare correlazioni significative fra le caratteristiche delle stazioni e la presenza/assenza di specie e/o comunità di rilevante interesse, con le seguenti finalità: colmare alcuni \textit{gap} (evidenziati dal Piano Regionale Agricolo Forestale ()PRAF) sulla conoscenza della distribuzione e consistenza di queste specie/gruppi di specie nel territorio in esame, in particolare nei SIC di recente designazione, come ad esempio il SIC  IT5130009 ``Tre Limentre-Reno''.

Il progetto, finanziato dalla Regione Toscana e da una serie di partner del territorio provinciale di Pistoia, ha individuato nei chirotteri uno dei gruppi più idonei a fungere da indicatore biologico dello stato di integrità degli ecosistemi. Essi presentano infatti alcune caratteristiche, quali la presenza di numerose specie ecologicamente esigenti, la diffusione molto ampia e  una contattabilità relativamente semplice. Il monitoraggio, oltre all’ampliamento delle conoscenze sulla distribuzione delle specie, ha permesso l’individuazione di numerose colonie, sia di svernamento, sia di riproduzione, di specie anche particolarmente interessanti quali il Miniottero (\emph{Miniopterus schreibersii}), tre rinolofi (\emph{Rhinolophus ferrumequinum}, \emph{R. euryale}, \emph{R. hipposideros})  e il Vespertilio smarginato (\emph{Myotis emarginatus}). Queste colonie sono state utilizzate come \textit{focal point} per il proseguimento del progetto nel suo secondo anno, e vengono continuamente monitorate, raccogliendo dati numerici, fenologici, e microclimatici. Inoltre lo stretto rapporto con le Amministrazioni Pubbliche e la sensibilizzazione dei cittadini hanno permesso di concretizzare appropriate azioni di conservazione e di educazione.
} %% remember to close the abstract text block brace!!