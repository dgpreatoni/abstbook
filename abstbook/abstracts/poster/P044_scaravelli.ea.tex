
% \abstitle{title here}
% \absauthors{names and superscripts for affiliations here}
% \absaddress{affiliations, starting each one with its superscripts, separate affiliations with a \break}
% \abstext{
% \index{author abbreviated name, to be placed in authors' index}
% \index{create an index entry for each author}
%  The abstract text
% }

%% Abstract title
\abstitle{\textit{Human dimension} delle colonie di grandi \textit{Myotis} in Alto Adige: lotta biologica e uso del guano}

%% Author names
\absauthors{D. \textsc{Scaravelli}$^1$, P. \textsc{Priori}$^2$, C. \textsc{Drescher}$^3$, E. \textsc{Ladurner}$^3$}

\absaddress{$^1$Dipartimento di Scienze Mediche Veterinarie, Università di Bologna, via Tolara di sopra 50, Ozzano Emilia. email: \url{dino.scaravelli@unibo.it}\break
$^2$Dipartimento di Scienze della Terra, della Vita e dell’Ambiente. Università degli Studi di Urbino Carlo Bo, Campus Scientifico, loc. Crocicchia, 61029 Urbino. email: \url{pamela.priori@uniurb.it}\break
$^3$Museo di Scienze Naturali di Bolzano, Via Bottai 1, 39100 Bolzano
}

%% Abstract text
\abstext{
%% Author names for index. State each author separately using \index{Doe J.}
\index{Scaravelli D.}
\index{Priori P.}
\index{Drescher C.}
\index{Laduener E.}
%% The actual abstract text goes here
La convivenza tra chirotteri ed umani spesso porta a conflitti di varia entità nei quali i primi risultano invariabilmente perdenti. L'ancora diffusa diffidenza verso questi animali e l’ignoranza sul loro reale valore ecologico porta spesso le comunità o i singoli individui ad affrontare negativamente la presenza di chirotteri negli edifici o comunque nei pressi degli spazi di pertinenza umana.

Qui si riporta il caso di un atteggiamento molto positivo creatosi in Alto Adige a proposito delle grandi colonie di \emph{Myotis myotis} e \emph{M. blythii} presenti in alcune chiese pienamente utilizzate. A Gargazzone e a Vezzano sono presenti rispettivamente circa 2500 e 1500 esemplari, nel primo caso accompagnati da circa 100 \emph{M. emarginatus}.

Nel sottotetto della chiesa di Gargazzone gli animali si posizionano o tra le assi del tetto o sul muro di sostegno per poi foraggiare nelle aree adiacenti durante la notte: in passato sono state studiate la frequentazione dei frutteti locali e la composizione della dieta, che consta sia di grandi coleotteri, sia di molti piccoli insetti, oltre che di una quota importante di insetti fitofagi. La presenza dei pipistrelli a Gargazzone non solo è ben accetta, ma è anche fonte di attrazione locale con momenti di incontro, divulgazione e osservazione dell’involo. Nel paese tutti concordano sull’importanza di preservare la colonia per il suo importante ruolo per la lotta biologica nei meleti locali.

Nel sottotetto della chiesa di Vezzano sono stati realizzati appositamente un rialzo per raccogliere l’abbondante guano che viene prodotto e un abbaino, specificatamente progettato, che ha sostituito recentemente la via d’uscita degli animali dalla torre campanaria, che causava diversi problemi di imbrattamento. L’opera ha avuto nel complesso una spesa sostenuta in egual misura dalla locale comunità e dalla provincia Autonoma di Bolzano. Anche qui la coscienza degli abitanti del luogo dell’importanza di salvaguardare la ``loro'' colonia è diffusa e molto sentita.
 
Sia a Gargazzone, ogni anno, e sia a Vezzano, ogni due, viene eseguita periodicamente la raccolta del guano presente che, grazie al laboratorio della sezione di Forlì dell’Istituto Zooprofilattico Sperimentale della Lombardia e dell’Emilia, è stato analizzato nel suo ruolo di possibile concime.

La composizione del guano ``maturo'' in entrambi i siti è similare, con le seguenti percentuali: Vezzano N 14.27; P 1.23 e K 1.297 e Gargazzone N 13.69; P 0.746 e K 0.975 con un pH rispettivamente di 6.6 e 6.4. Si tratta quindi di un concime che apporta un elevato livello di azoto e ha un eccellente potere ammendante in terrenti alcalini, ma soprattutto risulta molto indicato per incentivare la formazione di acidi umici grazie alla notevole massa organica che lo caratterizza. A detta dei locali i risultati del suo impiego sono ottimi, specialmente nell’orticoltura.

In questo senso, quindi, le grandi colonie sono divenute una presenza positiva non solo accettata ma adeguatamente protetta dalla popolazione locale che le considera un bene aggiunto della valle da tutelare.
} %% remember to close the abstract text block brace!!