
% \abstitle{title here}
% \absauthors{names and superscripts for affiliations here}
% \absaddress{affiliations, starting each one with its superscripts, separate affiliations with a \break}
% \abstext{
% \index{author abbreviated name, to be placed in authors' index}
% \index{create an index entry for each author}
%  The abstract text
% }

%% Abstract title
\abstitle{\emph{Trypanosoma cruzi livingstoneii} in \emph{Miniopterus schreibersii} new for Italy}

%% Author names
\absauthors{L. \textsc{Clément}$^1$, D. \textsc{Scaravelli}$^3$, P. \textsc{Priori}$^3$, P. \textsc{Christe}$^1$}

\absaddress{$^1$Department of Ecology and Evolution, University of Lausanne\break
$^2$Department of Veterinary Medical Sciences, University of Bologna, via Tolara di sopra 50, Ozzano Emilia, 40064 Italy, email: \url{dino.scaravelli@unibo.it}\break
$^3$Department of Earth, Life and Environmental Sciences, University of Urbino, Campus Scientifico, via Cà Le Suore 2, 61029 Urbino, Italy, email: \url{pamela.priori@uniurb.it}}

%% Abstract text
\abstext{
%% Author names for index. State each author separately using \index{Doe J.}
\index{Clément L.}
\index{Scaravelli D.}
\index{Priori P.}
\index{Christe P.}
%% The actual abstract text goes here
Trypanosomes are haematozoan flagellates parasites found in all continents and in all classes of vertebrates. They are generally transmitted by different haematophagus groups of insects. Trypanosomes found in bats are included in the human infective clade of \emph{Trypanosoma cruzi} which is also known to infect ever all others mammals and contains many subspecies. In humans, \emph{T. cruzi} is known to be the agent of Chagas disease, one of the most important problem of public health in South America.

In the framework of a larger research on haemoparasites in Italian bat species that recently detected the presence of a wide infection of \emph{Polychromophilus melanipherus} (Witsenburg et al. 2015) in \emph{Miniopterus schreibersi}, new investigations were done on the presence of parasite in \emph{M. schreibersii} thanks to classical and molecular methods.

Three drops of blood were collected from an interfemoral vein sting on sterile paper. Also a classical blood smear was prepared from 15 specimens for colony. In the laboratory, a PCR procedure was selected to compare 18S ribosomal RNA gene to \emph{T. cruzi} spp. from Genbank and \emph{Trypanosoma brucei} (African clade) was used as outgroup.

In the sampled \emph{M. schreibersii} infection of \emph{T. c. livingstoneii}  were found, a species described for the first time in 2013 in \emph{Rhinolophus landeri} in Mozambique (Lima et al. 2013). Prevalence in the checked colonies varies between 26.7\% and 46.7\%.

In Italy, according to Lanza (1999), there are records only for \emph{Trypanosoma vespertilionis} Battaglia, 1904, found in \emph{Myotis nattereri}, \emph{Nyctalus noctula}, \emph{Pipistrellus kuhlii} and  \emph{P. pipistrellus}.

This first record is referring to samples of \emph{M. schreibersii} collected in Emilia Romagna, Toscana and San Marino Republic.
 
The high specificity between trypanosomes and their potential bat hosts may indicate that the \emph{T. cruzi} clade seems to have some ancestral adaptation to bat parasitism. Additional phylogeographic analyses should not only continue to test this hypothesis but also to facilitate detailed modeling of historic bat movements and so provide insight into their current distribution.

The project is ongoing with the new season in order to identify potential vectors, ecological factors influencing the different prevalence recorded and other colonies to sample.\break

\vskip3mm
\begin{footnotesize}
\textbf{References}

Lanza B., 1999. I parassiti dei pipistrelli (Mammalia, Chiroptera) della fauna italiana. Monografie XXX, Museo reg. Sc. Nat. Torino, 318 pp.\par

Lima L., Espinosa-Alvarez O., Hamilton P.B., et al., 2013. \emph{Trypanosoma livingstonei}: a new species from African bats supports the bat seeding hypothesis for the \emph{Trypanosoma cruzi} clade. Parasites \& Vectors 6: 221.\par

Witsenburg F., Clément L., Dutoit L., Lòpez Baucells A., Palmeirim J., Pavlinic I., Scaravelli D., Ševcík M., Brelsford A., Goudet J., Christe P., 2015. How Malaria Gets Around: the Genetic Structure of a Parasite, Vector, and Host Compared. 16\textsuperscript{th} International Bat Research Conference \& 43 North American Symposium on Bat Research: 168-169. Molecular Ecology, 24 (4): 926–940.\par
\end{footnotesize}
} %% remember to close the abstract text block brace!!