% Abstract file structure example : 
% \abstitle{title here}
% \absauthors{names and superscripts for affiliations here}
% \absaddress{affiliations, starting each one with its superscripts, separate affiliations with a \break}
% \abstext{
% \index{author abbreviated name, to be placed in authors' index}
% \index{create an index entry for each author}
%  The abstract text
% }

%% Abstract title
\abstitle{The diet of Geoffroy's bat (\emph{Myotis emarginatus}) in an agriculture-dominated landscape (river Ticino valley - Lombardy)}

%% Author names
\absauthors{L. \textsc{Bruhat}$^{1,2}$, S. \textsc{Bologna}$^{1,3}$, M. \textsc{Spada}$^{1,3}$, A. \textsc{Molinari}$^{1,3}$, R. \textsc{Bettinetti}$^4$, E. \textsc{Boggio}$^4$, A. \textsc{Martinoli}$^1$, D. \textsc{Preatoni}$^1$}

\absaddress{$^1$Unità di Analisi e Gestione delle Risorse Naturali – \textit{Guido Tosi Research Group}, Dipartimento di Scienze Teoriche e Applicate, Università degli Studi dell’Insubria, Via J. H. Dunant 3, I - 21100 Varese, Italy\break
$^2$Aix Marseille Université, France\break
$^3$Istituto Oikos, Via Crescenzago, 1 Milano\break
$^4$Dipartimento di Scienze Teoriche e Applicate, Università degli Studi dell'lnsubria, Via Valleggio 11, 22100 Como, Italy}

%% Abstract text
\abstext{
%% Author names for index. State each author separately using \index{Doe J.}
\index{Bruhat L.}
\index{Bologna S.}
\index{Spada M.}
\index{Molinari A.}
\index{Bettinetti R.}
\index{Boggio E.}
\index{Martinoli A.}
\index{Preatoni D.}
%% The actual abstract text goes here
The River Ticino Valley is a green tongue through the Po Plain surrounded by a vast agricultural landscape, that hosts one of the biggest Geoffroy’s bat (\emph{Myotis emarginatus}) nursery in Europe. The definition and implementation of conservation measures for the species, listed in the Annexes II and IV of the Habitat Directive, are compulsory.

We analysed bat guano by collecting pellets under the colony on a weekly basis, from the beginning of May until the mid of July, in order to: (1) determine the trophic niche of the nursery; (2) obtain information about the foraging areas of the animals; (3) measure the presence of banned persistent pollutants in the guano.

Araneida made up for the main part of bat diet (freq\textsubscript{occurrence}=0.92\%; vol\textsubscript{tot}=50.34\%), followed by Homoptera (Cercopidae) (freq\textsubscript{occurrence}=0.52\%; vol\textsubscript{tot}=32.61\%) and Coleoptera (freq\textsubscript{occurrence}=0.34\%; vol\textsubscript{tot}=9.53\%). Large variations in diet composition were observed in June, with a reversal in the proportion of Homoptera (vol\textsubscript{tot}=52.8\% in June and 23.9\% in other periods), which made up the most of the diet, and Araneida (vol\textsubscript{tot}=27.3\% in June and 60.2\% in other periods).

The concentrations of organic persistent pollutant in the bat guano were low (DDT=6.34$\div$10.21 ng g\textsuperscript{-1} p.s.; PCB=33.77$\div$43.67 ng g\textsuperscript{-1} p.s.), indicating the persistence of a diffused contamination of banned substances in the area. These low concentrations suggest that the trophic resources exploited by bats were not significantly affected by persistent pollutant, maybe also thanking to the presence of a big biodynamic farm in the vicinities of the colony, but highlight the need of a specific monitoring scheme that can show the bioaccumulation of new pollutants that will be used in agriculture.
} %% remember to close the abstract text block brace!!