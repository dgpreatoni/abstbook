
% \abstitle{title here}
% \absauthors{names and superscripts for affiliations here}
% \absaddress{affiliations, starting each one with its superscripts, separate affiliations with a \break}
% \abstext{
% \index{author abbreviated name, to be placed in authors' index}
% \index{create an index entry for each author}
%  The abstract text
% }

%% Abstract title
\abstitle{Progetto LIFE Natura LIFE+08NAT/IT/000326 ``Fauna di Montenero''. Primi risultati delle azioni di conservazione sui chirotteri nel SIC ``Monte Calvo -- Piana di Montenero'' (Parco Nazionale del Gargano, Puglia, Italy)}
%% Author names
\absauthors{M. \textsc{Gioiosa}$^{1,2}$, P.P. \textsc{De Pasquale}$^1$}

\absaddress{$^1$Centro Studi Naturalistici Onlus – Foggia (Italy). email: \url{gioiosa@centrostudinatura.it}\break
$^2$Museo Provinciale di Storia Naturale – Foggia (Italy)}

%% Abstract text
\abstext{
%% Author names for index. State each author separately using \index{Doe J.}
\index{Gioiosa M.}
\index{De Pasquale P.P.}
%% The actual abstract text goes here
Il Parco Nazionale del Gargano è costituito da un mosaico di ambienti naturali: dalle faggete alle falesie marine. Le aree interne del Gargano sono ben rappresentate dal SIC ``Monte Calvo -- Piana di Montenero'', la cui morfologia fortemente carsica comprende doline (la maggiore concentrazione in Europa), grotte, grave, inghiottitoi, valli e campi carreggiati. La vegetazione è composta da boschi di latifoglie (cerrete, boschi di roverella, pochi castagneti e rare stazioni di Pioppo tremolo presenti sul fondo delle doline ad altitudine più elevata), prati pascoli steppici con affioramenti rocciosi e agroecosistemi variamente composti a formare un complesso intreccio di habitat che rendono questo SIC un vero scrigno di biodiversità in soli 7626 ettari.

Il progetto LIFE ``Fauna di Montenero'', cofinanziato al 50\% dall’UE, è stato coordinato dall'Ente Parco Nazionale del Gargano e realizzato in collaborazione con il Centro Studi Naturalistici Onlus e l'Azienda Agricola ``Montenero''.

Le azioni concrete di conservazione (Azioni C) costituiscono il ``cuore'' di un progetto LIFE. Le cinque Azioni C del LIFE Montenero, indirizzate alla conservazione di Anfibi, Rettili e Chirotteri, hanno permesso: il ripristino di 10 ``cutini'' (piccole raccolte d’acqua tradizionali, realizzate dall’uomo con muretti a secco sul fondo delle doline) per gli Anfibi (C1), la piantumazione di 10000 metri lineari di siepi vicino ai muretti a secco per i Rettili (C2), il posizionamento di 1000 bat box per i Chirotteri (C3), la realizzazione di un Centro di allevamento per Anfibi e Rettili (C4) e la messa in sicurezza di 9 grotte/inghiottitoi per i Chirotteri (C5).

Fondamentale per la verifica dell’efficacia delle azioni concrete di conservazione è stato il monitoraggio delle specie obiettivo (Azione E2). Le attività di monitoraggio chirotterologico, concluse a giugno 2015, e i cui dati sono in corso di elaborazione, hanno compreso l'esecusione di rilievi con varie tecniche (ispezione bat box, conteggi in grotte all’emergenza con videocamera IR; percorsi notturni di ascolto e rilievo bioacustico con bat detector). I risultati preliminari del monitoraggio delle due azioni riguardanti i chirotteri, sono di seguito sintetizzati:
\begin{compactdesc}
\item[Azione C3 Bat box]: come già risultava dai dati annuali, nei boschi oggetto di miglioramento si è registrata una maggiore attività notturna e un incremento delle specie (dati bioacustici). Inoltre la progressiva occupazione delle bat box, è passata da circa il 9\% del 2013 (dopo un anno dal posizionamento) a circa il 20\% del 2014, con un \textit{trend} che fa ben sperare per i prossimi anni. Tra le specie che hanno utilizzato i rifugi sono state osservate direttamente principalmente \emph{Myotis myotis} e \emph{Nyctalus lasiopterus} e in una occasione \emph{Myotis bechsteinii};
\item[Azione C5 Grotte]: incremento (specie/individui) delle colonie nelle grotte migliorate o colonizzazione di quelle ove l’accesso era precluso. L’esempio più eclatante riguarda l’Abisso Cinese (PU2191 del Catasto regionale), grava il cui ingresso era ingombro di pneumatici, in quanto già a poche settimane dalla messa in sicurezza, sono stati conteggiati 31 esemplari (contro 0--1 esemplari registrati prima dell’intervento). Anche il buon numero di specie rilevate (n=7), segnala il successo dell’azione, le specie osservate sono state: \emph{Rhinolophus ferrumequinum}, \emph{Rhinolophus hipposideros}, \emph{Rhinolophus euryale}, \emph{Myotis emarginatus}, che sono spiccatamente troglofile (oltre a \emph{Myotis} non id.), a queste si aggiungono \emph{Pipistrellus pipistrellus}, \emph{Pipistrellus kuhlii} e \emph{Hypsugo savii} che non sempre sono associate alle cavità e quindi potrebbero anche essere state registrate nei pressi delle grotte in attività di foraggiamento (nell’area sono presenti bat box).
\end{compactdesc}

Il monitoraggio ha inoltre consentito di documentare l’utilizzo, anche da parte dei chirotteri, dei cutini quali importanti siti di foraggiamento (maggiore presenza di insetti). I cutini quindi, in qualità di attrattori trofici della chirotterofauna, hanno dimostrato un valore aggiunto non atteso rispetto a quello, previsto, riguardante gli Anfibi. È evidente e importante la sinergia tra le azioni C3 (bat box) e C5 (grotte) con la C1 (cutini) in termini di miglioramento integrato della nicchia ecologica di diverse specie di chirotteri (forestali e troglofili). Infatti, ove queste azioni sono localizzate in aree relativamente vicine, mentre le prime due azioni vanno a migliorare il rifugio, la terza ne potenzia la nicchia trofica. A titolo di esempio una delle osservazioni più interessanti di Barbastello (\emph{Barbastella barbastellus}), è stata effettuata proprio su un cutino ripristinato, in un area poco idonea alla specie (che è legata alle foreste vetuste) ma dove, nelle vicinanze, si spera possa aver utilizzato una delle bat box installate con il progetto.

Ben 18 le specie rilevate, di cui 7 (in grassetto nella seguente \textit{checklist}), nuove per il SIC.
\emph{Rhinolophus ferrumequinum}, \emph{Rhinolophus hipposideros}, \emph{Rhinolophus euryale}, \emph{Myotis capaccinii}, \emph{Myotis myotis}, \emph{Myotis myotis/blythii}, \emph{\textbf{Myotis emarginatus}}, \emph{\textbf{Myotis nattereri}}, \emph{\textbf{Myotis bechsteinii}}, \emph{Pipistrellus kuhlii}, \emph{\textbf{Pipistrellus pipistrellus}}, \emph{\textbf{Pipistrellus pygmaeus}}, \emph{\textbf{Nyctalus leisleri}}, \emph{Hypsugo savii}, \emph{Eptesicus serotinus}, \emph{\textbf{Barbastella barbastellus}}, \emph{Miniopterus schreibersii}, \emph{Tadarida teniotis}.
} %% remember to close the abstract text block brace!!