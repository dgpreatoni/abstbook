% Abstract file structure example : 
% \abstitle{title here}
% \absauthors{names and superscripts for affiliations here}
% \absaddress{affiliations, starting each one with its superscripts, separate affiliations with a \break}
% \abstext{
% \index{author abbreviated name, to be placed in authors' index}
% \index{create an index entry for each author}
%  The abstract text
% }

%% Abstract title
\abstitle{Monitoring and mapping the distribution of Ireland's bat species}

%% Author names
\absauthors{G. \textsc{Giacomini}$^1$, T. \textsc{Aughney}$^2$, N. \textsc{Roche}$^2$}

\absaddress{$^1$Science and Management of Nature Master, School of Science, University of Bologna, Italy, email: \url{giada.giacomini@studio.unibo.it}\break
$^2$Bat Conservation Ireland \url{www.batconservationireland.org}}

%% Abstract text
\abstext{
%% Author names for index. State each author separately using \index{Doe J.}
\index{Giacomini G.}
\index{Aughney T.}
\index{Roche N.}
%% The actual abstract text goes here
There are nine resident bat species in Ireland. The Irish Bat Monitoring Programme is managed by Bat Conservation Ireland and consists of four schemes monitoring seven Irish bat species and one distribution survey (BATLAS 2020) collecting data on all nine Irish bat species. The longest running programme (since the 1980s) is the Lesser Horseshoe Roost Monitoring Scheme and involves the counting of \emph{Rhinolophus hipposideros} in both winter and summer counts and is principally undertaken by National Parks and Wildlife Service (NPWS) regional staff and Vincent Wildlife Trust. This bat species is found in the six western seaboard counties of Mayo, Galway, Clare, Limerick, Kerry and Cork. Summer surveys are completed principally using emergence counts while winter counts are completed by counting bats internally. The Car-based Bat Monitoring Scheme has been running since 2003. For this scheme volunteers drive known routes in 28 locations across the island (Republic of Ireland and Northern Ireland) and record all bat sounds along the roadside using time-expansion bat detectors. The survey takes place in July and August. Bat sounds are analysed by Bat Conservation Ireland after the survey has been completed. The species monitored using this scheme are the common pipistrelle \emph{Pipistrellus pipistrellus}, soprano pipistrelle \emph{P. pygmaeus} and Leisler's bat 
\emph{Nyctalus leisleri}. The third scheme, running since 2006, relies upon citizen scientists surveying 1 km transects along rivers and canals for Daubenton’s bats \emph{Myotis daubentonii} throughout the island. This scheme is very popular with people new to using bat detectors and is a great opportunity for people to receive free training and a loan of a bat detector for the summer months. In excess of 200 waterway sites are surveyed annually. Survey volunteers also collate additional information on habitats and the presence/absence of street lighting. From such data we have shown that there is an 11\% reduction in the activity of this species along waterways where street lights are present. The final monitoring scheme, in operation since 2007, counts brown long-eared bats \emph{Plecotus auritus} within roosts. A selection of fifty maternity roosts across the Republic of Ireland only are surveyed annually. Roosts monitored by emergence counts are completed three times during the summer months and start 20 minutes after sunset. For those roosts monitored using internal counts, two surveys are completed. Statistical analysis has shown that emergence counts are more reliable and provide more robust data for trend analysis. BATLAS 2020 is a follow up to BATLAS 2010. BATLAS 2010 project vastly increased our knowledge of bats throughout the island and filled in many gaps where no bats had been recorded for the first decade of the 21\textsuperscript{st} century. Every 10 km square (>900 10 km squares) will be re-surveyed to re-map the distribution of Ireland’s bats. As part of this programme, additional investigations will be undertaken to design a monitoring scheme for the two remaining Irish bat species: Natterer’s bat \emph{Myotis nattereri} and Whiskered bat \emph{M. mystacinus}.

As an Erasmus student I had the possibility to join Bat Conservation Ireland. I got involved in setting up and checking volunteers equipment prior to summer survey, assisted in training courses and completed roosts monitoring surveys at both lesser horseshoe bat and brown long-eared bat roosts. In addition, during my exchange, I was charged with two projects. The first involved visiting established bat box schemes and checking their usage by bats. All of the data collated was entered onto the Bat Conservation Ireland database. The second project was to pilot the monitoring of Nathusius’ pipistrelle usage of lakes using static bat detectors. In my opinion the monitoring schemes in Ireland are well organized  and represent the bat populations of the whole island and  I find that the volunteers are really enthusiastic and prepared which guarantees excellent results.
} %% remember to close the abstract text block brace!!