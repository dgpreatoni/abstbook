% Abstract file structure example : 
% \abstitle{title here}
% \absauthors{names and superscripts for affiliations here}
% \absaddress{affiliations, starting each one with its superscripts, separate affiliations with a \break}
% \abstext{
% \index{author abbreviated name, to be placed in authors' index}
% \index{create an index entry for each author}
%  The abstract text
% }

%% Abstract title
\abstitle{Ai chirotteri il riso piace bio!}

%% Author names
\absauthors{R. \textsc{Toffoli}}

\absaddress{Chirosphera Associazione per lo studio e la tutela dei chirotteri e l’ambiente. email: \url{chirosphera@gmail.com}}

%% Abstract text
\abstext{
%% Author names for index. State each author separately using \index{Doe J.}
\index{Toffoli R.}
%% The actual abstract text goes here
L’intensificazione agraria ha avuto gravi effetti negativi sulla biodiversità e può essere considerata una delle principali cause di riduzione dei Chirotteri in Europa durante la seconda metà del ventesimo secolo. In particolare l’incremento dell’uso di prodotti chimici ha ridotto drasticamente le disponibilità di cibo determinando una diminuzione delle popolazioni di molte specie nelle aree più intensamente coltivate.
 
L’utilizzo di pratiche agronomiche di tipo biologico costituisce uno strumento importante per aumentare la disponibilità di prede e aree di foraggiamento negli agrosistemi. L’attività dei Chirotteri risulta, infatti, maggiore in coltivazioni biologiche, dove non viene fatto uso di pesticidi, rispetto a quelle convenzionali. 

Questo studio mette in evidenza la differente frequentazione dei Chirotteri in risaie biologiche rispetto a quelle convenzionali. Sono state indagate quattro aziende risicole in provincia di Vercelli (Piemonte, Italia Nord Occidentale), di cui due biologiche e due convenzionali. L’attività dei Chirotteri è stata misurata mediante bat detector automatici SM2BAT+ e Elekon Batlogger posizionati al centro della superficie coltivata e attivi dal tramonto all’alba nel periodo compreso tra maggio e giugno 2015 per un totale di 30 ore. Sono state registrate 6526 sequenze acustiche relative a 13 taxa di cui 7 determinati a livello specifico (\emph{Eptesicus serotinus}, \emph{Hypsugo savii}, \emph{Nyctalus leisleri}, \emph{Pipistrellus kuhlii}, \emph{Pipistrellus nathusii}, \emph{Pipistrellus pipistrellus} e \emph{Pipistrellus pygmaeus}).

Nelle risaie biologiche sono state registrate l’86.7\% delle sequenze acustiche e il 98.0\% dei \textit{feeding buzz} rilevati evidenziando nel complesso una significativa maggiore attività di volo e di foraggiamento rispetto alle risaie convenzionali (attività di volo: t=4.460 df=16, p=0.0004; \textit{feeding buzz}: t=5.199, df=16, p<0.0001). \emph{Eptesicus serotinus}, \emph{Pipistrellus kuhlii} e \emph{Pipistrellus pipistrellus} hanno mostrato una attività maggiore nelle risaie biologiche (\emph{Eptesicus serotinus}: t=3.307 df=16, p=0.0045; \emph{Pipistrellus kuhlii}: t=6.667 df=16, p<0.0001; \emph{Pipistrellus pipistrellus}: t=4.066 df=16, p=0.0009), mentre le altre specie non hanno evidenziato differenze significative. \emph{Pipistrellus pygmaeus} e \emph{Plecotus} sp. hanno frequentato in maniera esclusiva le risaie biologiche.

Questi risultati, anche se preliminari, mostrano come le coltivazioni biologiche rappresentino importanti aree di foraggiamento negli ambienti agrari, in particolare nelle risaie dove i Chirotteri possono contribuire efficacemente al contenimento di alcuni insetti  antagonisti, in particolare la piralide del riso \emph{Chilo suppressalis}, riducendo il danno alle coltivazioni. 
} %% remember to close the abstract text block brace!!