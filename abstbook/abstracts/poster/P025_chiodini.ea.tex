% Abstract file structure example : 
% \abstitle{title here}
% \absauthors{names and superscripts for affiliations here}
% \absaddress{affiliations, starting each one with its superscripts, separate affiliations with a \break}
% \abstext{
% \index{author abbreviated name, to be placed in authors' index}
% \index{create an index entry for each author}
%  The abstract text
% }

%% Abstract title
\abstitle{Monitoraggio delle colonie di Chirotteri della Liguria}

%% Author names
\absauthors{E. \textsc{Chiodini}$^1$, F. \textsc{Oneto}$^2$, C. \textsc{Spilinga}$^1$, D. \textsc{Ottonello}$^2$, E. \textsc{Bertone}$^3$}

\absaddress{$^1$Studio Naturalistico Hyla s.n.c. - Via Aganoor Pompili, 4 - 06069 Tuoro sul Trasimeno (PG)\break
            $^2$Ce.S.Bi.N s.r.l c/o DISTAV - Università di Genova Corso Europa, 26 - 16132 Genova (GE)\break  
            $^3$Parco Alpi Liguri c/o Comune di Pigna Piazza Umberto I – 18037 Pigna (IM)}

%% Abstract text
\abstext{
%% Author names for index. State each author separately using \index{Doe J.}
\index{Choidini E.}
\index{Oneto F.}
\index{Spilinga C.}
\index{Ottonello D.}
\index{Bertone E.}
%% The actual abstract text goes here
La ricerca, inserita nell’ambito del ``\textit{POR Liguria FESR 2007/2013 ASSE 4, linea di attività 4.2 Valorizzazione e fruizione Rete Natura 2000}'', ha previsto il monitoraggio nel periodo settembre 2013 - settembre 2015 di alcune colonie di particolare interesse conservazionistico presenti nel territorio ligure. L’indagine ha interessato complessivamente 73 siti riconducibili alle seguenti tipologie di \textit{roost}: cavità naturali, cave e miniere dismesse, bunker e gallerie artificiali, edifici storici e ruderi. I siti sono stati ispezionati in periodo invernale ed estivo al fine di verificare l’occupazione degli stessi da parte dei Chirotteri, la tipologia e la consistenza delle colonie presenti.

Durante i rilievi invernali, la chirotterofauna ibernante è stata censita mediante conteggio visivo diretto degli esemplari o foto, mentre i rilievi estivi sono stati condotti applicando diverse metodologie in funzione delle caratteristiche della colonia interessata e della tipologia di \textit{roost}.
 
In particolare si è proceduto con i seguenti metodi di censimento: visivo diretto all'interno del \textit{roost}; da foto all'interno del \textit{roost}; visivo diretto degli esemplari in transito attraverso l’accesso al \textit{roost}; da ripresa video degli esemplari in transito attraverso l’accesso al \textit{roost}.
 
Se necessario sono state inoltre previste catture degli individui mediante \textit{mistnet} o \textit{harp trap} a seconda delle caratteristiche del sito.

Il monitoraggio ha permesso di fornire un quadro aggiornato su presenza, distribuzione e stato conservazionistico delle principali colonie note presenti nel territorio regionale.

Tra i \textit{roost} di maggior interesse monitorati, figura la Grotta di Bocca Lupara (SP), sito di rilevanza nazionale che ospita in periodo estivo un’importante colonia di \emph{Miniopterus schreibersii}; l’Arma do Principà (SV), sito riproduttivo di \emph{Myotis oxygnathus}; l’Arma della Pollera (SV), noto sito di svernamento di \emph{Rhinolophus ferrumequinum}; alcune miniere abbandonate in Val Graveglia (GE), il cui sistema di gallerie viene sfruttato dal rinolofo maggiore e la Galleria di Glori (IM), utilizzata come \textit{roost} invernale da \emph{Rhinolophus euryale}, \emph{Rhinolophus ferrumequinum} ed in misura minore da \emph{Rhinolophus hipposideros}.
 
Nell’estate 2014 sul soffitto di quest’ultima galleria è stata inoltre rilevata una colonia di rinolofidi che non è stato possibile determinare a causa dell’elevata altezza del soffitto da terra, e pertanto saranno approntati ulteriori indagini nel proseguo del monitoraggio.

Relativamente a \emph{Rhinolophus hipposideros}, sono note infine alcune importanti colonie riproduttive all’interno di edifici religiosi, tra cui il Santuario Nostra Signora della Montà presso il cimitero di Molini di Triora (IM) e la Chiesa della Madonna di Pallara (IM).
} %% remember to close the abstract text block brace!!