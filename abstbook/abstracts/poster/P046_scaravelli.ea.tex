
% \abstitle{title here}
% \absauthors{names and superscripts for affiliations here}
% \absaddress{affiliations, starting each one with its superscripts, separate affiliations with a \break}
% \abstext{
% \index{author abbreviated name, to be placed in authors' index}
% \index{create an index entry for each author}
%  The abstract text
% }

%% Abstract title
\abstitle{Recupero Chirotteri a Rimini: dati dai primi anni di attività}

%% Author names
\absauthors{D. \textsc{Scaravelli}$^1$, R. \textsc{Boga}, E. \textsc{Santolini}$^2$}

\absaddress{$^1$Dipartimento di Scienze Mediche Veterinarie, Università di Bologna, via Tolara di sopra 50, Ozzano Emilia. email: \url{dino.scaravelli@unibo.it}\break
$^2$A.N.P.A.N.A., Rimini}

%% Abstract text
\abstext{
%% Author names for index. State each author separately using \index{Doe J.}
\index{Scaravelli D.}
\index{Boga R.}
\index{Santolini E.}
%% The actual abstract text goes here
Il recupero dei Chirotteri ha importanti ripercussioni non tanto nel numero di esemplari accolti e riportati in natura, quanto per il valore educativo, didattico e soprattutto scientifico dell’azione. Questi animali necessitano, in ambiente controllato, di un notevole sforzo per la loro cura e in tal senso la raccolta sistematica delle informazioni faunistiche, comportamentali e ovviamente cliniche è un aspetto fondamentale di questa attività. Inoltre, la diffusione di questi dati con pubblicazioni a varia periodicità e il mantenimento di un \textit{database} aggiornato, è altresì elemento prioritario nella gestione di un punto di accoglienza, e nella sua organizzazione e sostegno.

Per l’area riminese si riportano qui in sintesi i dati relativi a 172 ingressi avvenuti dal 2009 al 2014 negli anni di attività, gestiti a titolo del tutto volontario. Per ogni esemplare sono stati raccolti i dati di ingresso, di anamnesi e di cura, oltre che l’\textit{exitus}.

La specie maggiormente rappresentata è stata \emph{Hypsugo savii} con il 69\% dei casi, seguita da \emph{Pipistrellus kuhlii} con il 29\% e singoli arrivi di \emph{P. pipistrellus} e \emph{Eptesicus serotinus}, oltre a 2 \emph{Pipistrellus} non identificabili. Il numero di casi ammessi è aumentato con il diffondersi dell’informazione sulla presenza di un centro di accoglienza, e si è passati dagli iniziali 15 casi nel 2009 agli 82 del 2014.
Le province di provenienza sono soprattutto Rimini (73\%), Forlì-Cesena (13.5\%), Pesaro-Urbino 5\%, con pochi esemplari provenienti da Repubblica di San Marino, Ancona e  Ravenna.

Le cause di ricovero sono state le seguenti: caduta dalla nursery (26.7\%), giovane post svezzamento debilitato (16.7\%), fratture ossee (13.3\%), debilitazione generale (11.3\%), infestazioni ectoparassitarie (10.7\%), attacco da gatto (8\%), lacerazioni (6\%) infezioni (4\%) e esemplari ritrovati dopo distruzioni dei rifugi o a terra (4\%).

Infine sono discussi alcuni casi clinicamente particolari e la gestione degli esemplari non più reinseribili in natura.
} %% remember to close the abstract text block brace!!
