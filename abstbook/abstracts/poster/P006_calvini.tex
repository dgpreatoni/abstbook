% Abstract file structure example : 
% \abstitle{title here}
% \absauthors{names and superscripts for affiliations here}
% \absaddress{affiliations, starting each one with its superscripts, separate affiliations with a \break}
% \abstext{
% \index{author abbreviated name, to be placed in authors' index}
% \index{create an index entry for each author}
%  The abstract text
% }

%% Abstract title
\abstitle{Dati storici e attuali sulla Chirotterofauna ipogea ligure: un confronto possibile?}

%% Author names
\absauthors{M. \textsc{Calvini}}

\absaddress{Via Dante Alighieri, 426 – 18038 Sanremo (IM). \url{mara.calvini@gmail.com}}

%% Abstract text
\abstext{
%% Author names for index. State each author separately using \index{Doe J.}
\index{Calvini M.}
%% The actual abstract text goes here
In Liguria sono attualmente segnalate 25 specie di Chirotteri. Di queste, almeno 14 (\emph{Rhinolophus ferrumequinum}, \emph{R. hipposideros}, \emph{R. euryale}, \emph{Myotis capaccinii}, \emph{M. myotis}, \emph{M. blythii}, \emph{M. mystacinus}, \emph{M. nattereri}, \emph{M. emarginatus}, \emph{Pipistrellus pipistrellus}, \emph{Plecotus auritus}, \emph{P. austriacus}, \emph{Barbastella barbastellus} e \emph{Miniopterus schreibersii}) frequentano gli ambienti ipogei, naturali e artificiali, in periodo invernale o estivo.

Le informazioni storiche a disposizione sulla Chirotterofauna delle cavità liguri derivano da una raccolta di dati relativi all’attività speleologica condotta principalmente nel periodo 1950--1980, per quanto si hanno notizie già dalla fine del 1800. 

Studi specifici mirati alla ricerca dei Chirotteri in grotta sono iniziati a partire dall’anno 2000.
Nel periodo storico (1950--1980) erano state rilevate 12 specie, per 11 delle quali è stata confermata la presenza a partire dal 2000 durante il monitoraggio di numerose cavità sotterranee a livello regionale.

Le diverse metodologie di campionamento e la maggiore standardizzazione nella raccolta dei dati in anni recenti rende difficile un confronto tra la situazione della Chirotterofauna storica e quella attuale. Le uniche specie di cui è possibile fare un confronto sono le tre specie di Rinolofidi presenti in Liguria, in quanto le loro caratteristiche ecologiche permettono una più facile osservazione, identificazione e conteggio degli esemplari.

Storicamente, nell’arco della fenologia annuale, \emph{R. ferrumequinum} era presente nel 76.3\% dei siti indagati, \emph{R. hipposideros} nel 34\% e \emph{R. euryale} nel 20.6\%. Attualmente la presenza riscontrata è rispettivamente del 53.7\%, 52.4\% e 6.1\%.

È stata eseguita un'analisi dei cambiamenti a lungo termine nelle popolazioni svernanti dei tre Rinolofi per 9 siti monitorati annualmente nel periodo 2000--2013 in Liguria. Il \textit{trend} delle popolazioni mostra un apparente incremento medio annuo (+2\%) solo per \emph{R. ferrumequinum}.

Il presente lavoro analizza le differenze di distribuzione storiche e attuali delle tre specie di Rinolofi nella regione Liguria, definisce l’andamento delle popolazioni attuali e vuole porre le basi per una fattiva collaborazione tra il mondo speleologico e la ricerca chirotterologica.
} %% remember to close the abstract text block brace!!